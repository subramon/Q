\section{Debugging}
\label{Debugging}

Our experience has been that ease of debugging has often been made
subservient to bragging rights on performance. Q deals with this as follows.

\be
\item 
Since Q is implemented as a Lua package, the CLION IDE from IntelliJ 
with the Lua plugin allows interactive debugging sessions. 
\item Sharing sessions. 
We have embedded Q in an HTTP server and have provided a command line
interface.
Two
developers could issue commands to the server, the commands being serialized at the
server. 
\item ``print''-ing a variable is less useful when the variable is a long
  vector. What is needed instead is 
to query some {\it property} of the vector e.g., min, max, is it sorted. 
We can {\tt Q.save} a session from the IDE, restore it in another session in
read-only mode and, when done with diagnosis, continue from the IDE.
\item Modifying state. In addition to inspecting variables, it is often useful
  to modify them so as to continue the debugging session without having to
  start over. For example, say that we hit a breakpoint and discover  
  that we had forgotten to sort a vector or load a table. We 
  (i) {\tt Save} from the IDE
  (ii) {\tt Restore} in a parallel session 
  (iii) perform the modifications in that session 
  (iv) {\tt Save} and quit the session 
  (v) lastly, {\tt Restore} in the IDE.

\ee
