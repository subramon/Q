\subsection{Logistic Regression}

A key subroutine in logistic regression \cite{Hastie2009} is improving the 
estimates of the
coefficients, \(\beta\), using the Newton-Raphson algorithm.
In the course of this, we need to compute
\(\frac{1}{1 + e^{-x}}\) and 
\(\frac{1}{(1 + e^{-x})^2}\).
We use this to motivate the ``conjoin'' feature of \Q.
Performing common sub-expression elimination as in 
Figure~\ref{sub_expr} yields a 2x speedup.

\begin{figure}
\centering
\fbox{
\begin{minipage}{8 cm}
\centering
\verbatiminput{sub_expr.lua}
\caption{Common sub-expression elimination}
\label{sub_expr}
\end{minipage}
}
\end{figure}


\subsubsection{Lock step evaluation}
However, the code in Figure~\ref{sub_expr} has a subtle but
critical bug when the number of elements in \(x\) exceeds the chunk
size.  If we were to call {\tt eval()} on \(y\), then we would
end up consuming sucessive chunks of \(t_3\). Now, if we were to call
{\tt eval()} on \(z\), we would fail when requesting the first
chunk of \(t_3\), since it has {\bf not} been memo-ized. One solution
is to ensure that \(y\) and \(z\) are evaluated in lock-step, after they have
been created, as in Figure~\ref{lock_step}.
\begin{figure}
\centering
\fbox{
\begin{minipage}{8 cm}
\centering
\verbatiminput{lock_step.lua}
\caption{Lock step evaluation}
\label{lock_step}
\end{minipage}
}
\end{figure}

The problem with this solution is that the burden of lock-step evaluation falls
on the \Q\ programmer. This is remedied by the {\tt conjoin} function
(Figure~\ref{conjoin})

\begin{figure}
\centering
\fbox{
\begin{minipage}{8 cm}
\centering
\verbatiminput{conjoin.lua}
\caption{Conjoined Vectors}
\label{conjoin}
\end{minipage}
}
\end{figure}

