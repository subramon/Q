\section{Conclusion}
\label{end}

For some time now, data analysis has been straining to 
slip the surly bonds of relational databases and SQL, as evidenced by the proliferation of systems
inspired by \cite{mapreduce2004}. Nonetheless, as Santayana postulated and \cite{antihadoop2009} demonstrated,
those who don't remember the past are condemned to repeat it.
Despite this, we believe that the development of \Q, is {\bf not} an
extraordinary act of hubris. Instead, it is a judicious application of
well-understood techniques to a limited but meaningful domain, encompassing
the day to day activities of the data scientist, such as 
ingestion, validation, preparation, feature engineering, and model building.

The development of \Q\ has been motivated by understanding, as data 
science practitioners, of what is truly needed to be
effective. We have found that engaging the user at the right level of abstraction, rather than shielding them from
system internals, makes for simple, performant code. It has allowed for the
rapid deployment of highly performant analytical capabilities without racking up
enormous AWS bills.
At the same time, \Q's
parsimonious use of computing resources shows that the pursuit of a profit
motive is not inconsistent with a recognition that computing uses 
material and energy resources \cite{Schumacher,Limits2018} and that there are 
``planetary boundaries that must not be transgressed'' \cite{Rockstrom2009}.

