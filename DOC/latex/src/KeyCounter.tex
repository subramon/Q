\startreport{KeyCounter}
\reportauthor{Ramesh Subramonian}

\section{Introduction}
The KeyCounter is a hash table that is used to count the number of distinct
occurrences of a set of composite keys. We intend to extend it to handle other use cases
as well. 

\bd
\item [Key] Type is a struct of between 1 and 4 atomic values e..g,
\begin{verbatim}
typedef struct _keytype { 
  uint32_t key1;
  double key2;
  char key3[16];
  } keytype;
\end{verbatim}
\item [Value]
Consists of following. 
\be
\item \verb+ uint32_t count+ number of times a key can occur is \(\leq 2^{31}\) 
\item \verb+ uint32_t guid+ number of unique keys \(\leq 2^{31}\) 
\ee
\ed

\section{new}
\label{new}
To create a KeyCounter, we pass in 
\be
\item {\tt vecs} --- an indexed table of vectors, all of which should have the same length and the same chunk size
\item {\tt optargs} --- a table which allows us to over-ride default values for
\be
\item name of KeyCounter. Has no value other than for debugging
\item label of KeyCounter. If label is {\tt foo}, then we creare
\be
\item a directory {\tt foo} in \verb+~/Q/TMPL_FIXHASHMAP/KEY_COUNTER/+
\item a library {\tt libkcfoo.so} used by LuaJIT's {\tt ffi.load()}
\ee
\ee
\ee

Internal members of KeyCounter class are 
\be
\item vecs, vectors used to create it
\item is\_eor, whether all vectors have been fully consumed
\item sum\_count, number of items inserted
\item \ldots
\ee
\section{clone}
\label{clone}

When applied on a KeyCounter, returns a brand new empty KeyCounter.
Takes same arguments as Section~\ref{new} with the following differences
\be
\item The number and type of the table of {\tt vecs} must match the ones used to
create the original KeyCounter
\item can set name in optargs but not label
\ee

\section{next}
\label{next}
Consumes one chunk from each of the vectors passed in.
{\tt is\_eor} is set to true when nothing more to consume.

\section{eval}
\label{eval}
Calls Section~\ref{next} repeatedly until all vectors have reached {\tt eov}.
Note that all vectors must have same length.

\section{label}
\label{label}
Returns label of KeyCounter. This cannot be modified. 
Set in Section~\ref{new}

\section{name}
\label{name}
Returns name of KeyCounter. 

\section{set\_name}
\label{set_name}
If no argument, then name set to nil.
Else, must pass string which is used as new name

\section{size}
\label{size}
Returns size of hash table

\section{nitems}
\label{nitems}
Returns number of items in hash table. Always less than size.

\section{is\_eor}
\label{is_eor}
Returns {\tt is\_eor} --- {\tt true} when vectors have been fully consumed; {\tt
false} otherwise.

\section{sum\_count}
\label{sum_count}
Returns the number of insertions into the hashmap.

\section{get\_val}
\label{get_val}

Given an indexed table of scalars that represents a key, 
returns the key and value and position in the hash table where that key
occurred. 

\be
\item Number of scalars same as number of vectors used in Section~\ref{new}
\item Types of scalars must match types of vectors used in Section~\ref{new}
\item Special case when type is {\tt SC}. The length of the string must be
lesser than the width of the corresponding vector.
\ee

Returns are 
\be
\item {\tt key}, pointer to the key struct inside the hash table. 
\item {\tt keytype}, string that can be used to cast returned key to proper tyope
for access from Lua 
\item {\tt val}, pointer to the value struct inside the hash table. 
\item {\tt valtype}, string that can be used to cast returned val to proper tyope
for access from Lua 
\item {\tt bool is\_found}, indicating whether the key was found
\item {\tt uint32\_t where\_found}, which is where the key was  found
in the hash table. 
\ee

\section{condense}
\label{condense}
Input is a field which can be either 
\be
\item {\bf native} field. Return type is I4, ideally UI4. Options are 
\be
\item count (non-zero)
\item guid (non-zero)
\item idx
\ee
\item {\bf auxiliary} field . Options are 
\be
\item cum\_count. Type is I8, ideally UI8.
\item \ldots
\ee
\ee
Output is a vector. Creation of this vector is 
best explained with an example. Let us say that the hash table looks like
Table~\ref{hash_tbl_1}. Depending on the input, we get back one of the columns
of Table~\ref{condensed_hash_tbl_1}.
\begin{table}
\centering
\begin{tabular}{|l|l||l|l|l||l|} \hline \hline 
{\bf Key1 } & {\bf Key 2} & {\bf count} & 
{\bf guid} & {\bf idx} & {\bf cum\_count} \\ \hline \hline
--- & --- & --- & --- &  0 & 0  \\ \hline
  1 &  6  &  3  &  1  &  1 & 0  \\ \hline
  2 &  5  &  2  &  2  &  2 & 3  \\ \hline
--- & --- & --- & --- &  3 & 5  \\ \hline
  3 &  4  &  1  &  3  &  4 & 5  \\ \hline
  4 &  3  &  6  &  6  &  5 & 6  \\ \hline
  5 &  2  &  5  &  5  &  6 & 12  \\ \hline
--- & --- & --- & --- &  8 & 17  \\ \hline
  6 &  1  &  4  &  4  &  8 & 17  \\ \hline
\hline
\end{tabular}
\caption{Sample Hash Table}
\label{hash_tbl_1}
\end{table}

\begin{table}
\centering
\begin{tabular}{|l|l|l||l|} \hline \hline 
count & guid & idx & cum\_count \\ \hline \hline
  3  &  1  &  1 & 0 \\ \hline
  2  &  2  &  2 & 3 \\ \hline
  1  &  3  &  4 & 5 \\ \hline
  6  &  6  &  5 & 6 \\ \hline
  5  &  5  &  6 & 12 \\ \hline
  4  &  4  &  8 & 17 \\ \hline
\hline
\end{tabular}
\caption{Condensation of Table~\ref{hash_tbl_1}}
\label{condensed_hash_tbl_1}
\end{table}

\section{make\_cum\_count}
\label{make_cum_count}

Makes an auxiliary field alongside the native fields {\tt count}
and {\tt guid}. This is creates from {\tt count}, example in 
Table~\ref{condensed_hash_tbl_1}.

\section{make\_permutation}
\label{make_permutation}
Input is an indexed table of vectors which must match the vectors used to create
the KeyCounter. Output is a vector of type I4, ideally UI4.

Best explained with an example. 
Consider hash table in Table~\ref{hash_tbl_2} with 4 items and a 
cumulative count of 10. Assume the input to Section~\ref{make_permutation} is in
Table~\ref{data_1}. 

\begin{table}
\centering
\begin{tabular}{|l|l||l|l|l||l|} \hline \hline 
{\bf Key1 } & {\bf Key 2} & {\bf count} & {\bf guid} \\ \hline \hline
--- & --- & --- & ---  \\ \hline
  D &  4  &  1  &  4   \\ \hline
--- & --- & --- & ---  \\ \hline
  C &  3  &  2  &  3   \\ \hline
  A &  1  &  4  &  1   \\ \hline
  B &  2  &  3  &  2   \\ \hline
--- & --- & --- & ---  \\ \hline
\hline
\end{tabular}
\caption{Sample Hash Table --- 2}
\label{hash_tbl_2}
\end{table}

\begin{table}
\centering
\begin{tabular}{|l|l||l|} \hline \hline 
{\bf Key1 } & {\bf Key 2} & {\bf Permutation} \\ \hline \hline
A & 1 & 3 \\ \hline
B & 2 & 7 \\ \hline
C & 3 & 1 \\ \hline
D & 4 & 0 \\ \hline
A & 1 & 4 \\ \hline
B & 2 & 8 \\ \hline
C & 3 & 2 \\ \hline
A & 1 & 5 \\ \hline
B & 2 & 9 \\ \hline
A & 1 & 6 \\ \hline
\hline
\end{tabular}
\caption{Data For Permutation}
\label{data_1}
\end{table}
