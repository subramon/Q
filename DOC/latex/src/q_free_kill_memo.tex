\documentclass[letterpaper,12pt]{article}
\usepackage{geometry}
 \geometry{
 a4paper,
 total={170mm,257mm},
 left=20mm,
 top=20mm,
 right=20mm,
 bottom=20mm,
 }
\usepackage{courier}
\usepackage{fancyheadings}
\pagestyle{fancy}
\usepackage{graphicx}
\setlength\textwidth{6.5in}
\setlength\textheight{8.5in}
\newtheorem{problem_statement}{Problem Statement}
\newtheorem{invariant}{Invariant}
\newcommand{\TBC}{\framebox{\textbf{TO BE COMPLETED}}}
\newtheorem{assumption}{Assumption}
\newcommand{\be}{\begin{enumerate}}
\newcommand{\ee}{\end{enumerate}}
\newcommand{\bi}{\begin{itemize}}
\newcommand{\ei}{\end{itemize}}
\newcommand{\bd}{\begin{description}}
\newcommand{\ed}{\end{description}}
\newtheorem{notation}{Notation}
\begin{document}
\title{Resource Reduction in Q}
% \author{ Ramesh Subramonian }
\maketitle
\thispagestyle{fancy}
\lhead{}
\chead{}
\rhead{}
% \lfoot{{\small Decision Sciences Team}}
\cfoot{}
\rfoot{{\small \thepage}}

\abstract{This document lists various techniques used to reduce Q's resource
consumption (in terms of memory and disk usage).
}

\section{Introduction}

\subsection{Motivation}
Since Q is an extension of Lua (LuaJIT to be precise), it performs garbage
collection. This allows it to free resources that are no longer needed.

However, there are two weaknesses with this approach. 
\be
\item Lua is unable to look inside a Vector to assess how much 
memory/disk it is using and therefore whether garbage collection should 
be triggered. 
\item Even if it could, there are many  situations where the Q programmer can
use their knowledge of the computation to free resources earlier than they would
have been freed had one relied solely on garbage collection. This is consistent
with Q's overall design philosophy of creating hooks that allow the 
advanced programmer to gain efficiency at the cost of thinking harder.
\ee

\subsection{Organization}
We provide the Q programmer with three mechanisms to inform the Q runtime 
that a vector can be freed in total or in part. When using these 
mechanisms, we impose some constraints on how they can be used not for 
any intrinsic reason but because of my general belief that one should 
provide the system with as few degrees of freedom without 
hamstringing the Q programmer. 

\be
\item memo --- Section~\ref{memo}
\item kill --- Section~\ref{kill}
\item early free --- Section~\ref{early_free}
\ee

\subsection{Data Structures}
See Table~\ref{tbl_data_structures}

\begin{table}
  \centering
  \begin{tabular}{|l|l|l|l|l|} \hline \hline 
    {\bf Name in \LaTeX} & {\bf In} {\tt q\_cfg.lua} & {\bf C Type} & {\bf C variable} & {\bf Default} \\ \hline \hline 
    \(n_E\) & & uint64\_t & num\_elements & \\ \hline 
    \(p\) & & bool & is\_persist & false \\ \hline
    \(e\) & & bool & is\_eov & false \\ \hline
     \(c_{\mathrm{min}}\) & & uint32\_t & min\_chnk\_idx & \\ \hline 
     \(c_{\mathrm{max}}\) & & uint32\_t & max\_chnk\_idx & \\ \hline 
    \(m\) & Y & bool & is\_memo & false \\ \hline
    \(n_M\) & Y & uint32\_t & memo\_len & 0 \\ \hline 
    \(k\) & Y & bool & is\_killable & false \\ \hline
    \(n_K\) & Y & uint32\_t & num\_kill\_ignore & 0 \\ \hline 
    \(f\) & Y & bool & is\_freeable & false \\ \hline
    \(n_F\) & Y & uint32\_t & num\_free\_ignore & 0 \\ \hline 
    \hline
  \end{tabular}
  \caption{Data Structures for Early Resource Allocation}
  \label{tbl_data_structures}
\end{table}

\subsection{Default Values}
Default values for properties marked with a Y in Table~\ref{tbl_data_structures}
are set in \verb+~/Q/UTILS/lua/qcfg.lua+
They can be changed using \verb+qcfg._modify()+ 
\subsection{Persistence}
A vector will {\bf not} be persisted unless \(m = k = f = \mathrm{false}\).

\subsection{When can changes be made}
\subsubsection{memo}
\label{memo_change}
\(m\) can be set to true only if 
\be
\item \(n_E = 0\)
\item \(f = \mathrm{false}\)
\item \(p = \mathrm{false}\)
\ee
\subsubsection{kill}
\label{kill_change}
\(k\) can be set to true only if 
\be
\item \(n_E = 0\)
\item \(p = \mathrm{false}\)
\ee

\subsubsection{free}
\label{free_change}

\(f\) can be set to true only if 
\be
\item \(n_E = 0\)
\item \(p = \mathrm{false}\)
\item \(m = \mathrm{false}\)
\ee
%-------------------------------------------------------------
\section{Memo}
\label{memo}

If we do not ``memo'' the vector, then it holds on to all chunks created.
We use memo to indicate the maximum number of chunks the vector should hold on
to. For example, say we set the memo len to 2. 
When we create the first chunk, the vector holds on to chunk \(\{0\}\). 
When we create the second chunk, the vector holds on to chunks \(\{0, 1\}\).
When we create the third chunk, the vector discards the \(0^{th}\) chunk and
holds on to chunks \(\{1, 2\}\)


\subsection{Invariants}
\begin{invariant}
\label{m_1}
\(m = \mathrm{false} \Rightarrow n_M = 0\)
\end{invariant}

\begin{invariant}
\label{m_2}
\(m = \mathrm{true} \Rightarrow n_M > 0\)
\end{invariant}

For a Vector that has been memo'd, the number of chunks cannot exceed \(n_M\).
Hence, Invariant~\ref{m_3}
\begin{invariant}
\label{m_3}
  \(m = \mathrm{true} \Rightarrow c_{\mathrm{max}} - c_{\mathrm{min}} + 1 \leq n_M\)
\end{invariant}

\section{Kill-able}
\label{kill}

Killable was invented to handle the case when we know that a vector will no
longer be needed. For example, assume \(x \leftarrow y + z\) and we know that
once \(x\) has been computed, there is no further need for \(y\). In that
case, the creator of \(x\) will issue a kill signal to \(y\). If \(y\) was
marked killable, then it is deleted and all its resources freed. 

To complicate matters, when we set a vector as killable, we set the number of
lives that it has. In the above example, assume that \(y\) was needed not just
for the creation of \(x\) but also for the creation of \(w \leftarrow \sum y\).
In that case, we would set \(n_K(y) \leftarrow 1\). Note that the creator of \(x\) and
the creator of \(w\) both issue kill signals to \(y\). 

When \(y\) receives the first kill signal, 
it is ignored because \(n_K(y) \neq 0\). However, we decrement
\(n_K(y)\) by 1 , setting it to 0 in this example.
When \(y\) receives the second kill signal, 
it is not ignored because now \(n_K(y) = 0\) and \(y\) is deleted.

As usual, if \(k(y) = \mathrm{false}\), then kill signals received by \(y\) are ignored.

\subsection{Invariants}
\begin{invariant}
\(k = \mathrm{false} \Rightarrow n_K = 0\)
\end{invariant}

Questions
\be
\item Can you kill a vector that is not yet eov?
\ee

\section{Free-able}
\label{early_free}

There are situations (e.g., the {\tt where} operator) where we do not 
in advance how many chunks we need to remember. However, we do know 
that if the vector receives a ``free'' signal, then it can delete all but the
most recent chunk. We require the {\tt early\_free()} call to provide a chunk index.
So, {\tt early\_free(\(i\))} means that chunks \(\{0, 1, 2 \ldots i-1\}\) 
will be deleted.  

Unfortunately, the situation gets more complicated, akin to Section~\ref{kill},
because a vector may have more than one consumer. Setting \(f \leftarrow
\mathrm{true}\) requires us to set \(n_F \geq 0\). Assume \(n_F > 1\). When a
chunk is created, we assign it a ``free-life'' of \(n_F\). When,
\(x:\mathrm{early\_free}(i)\) is invoked, each chunk of \(x\) whose index is \(< i\) is
examined. If its free life is 0, it is deleted. If its free life is \(> 0\),
then its free life is decremented by 1.

As usual, if \(f(y) = \mathrm{false}\), then early free signals received by \(y\) are ignored.

You cannot use memo and freeable at the same time. 
Hence, Invariants~\ref{f_3}, ~\ref{f_4}.

\begin{invariant}
\label{f_3}
  \(m = \mathrm{true} \Rightarrow f = \mathrm{false}\)
\end{invariant}

\begin{invariant}
\label{f_4}
  \(f = \mathrm{true} \Rightarrow m = \mathrm{false}\)
\end{invariant}

\section{Actions}

There are three kinds of actions, classified as 
\be
\item do  --- performs the action.
  \be
\item {\tt kill} e.g., \(x:\mathrm{kill}\) where \(x\) is a vector 
\item {\tt early\_free} e.g., 
  \(x:\mathrm{early\_free}(i)\) where \(x\) is a vector and \(i\) is a chunk index
\item {\tt memo} --- we have chosen to {\bf not} expose this ability 
  from Lua. It is available from C. It is not meant to be invoked by the Q
  programmer. 
  \ee
\item is --- checks the specified property
  \be
\item {\tt is\_early\_free}
\item {\tt is\_memo\_len}
\item {\tt is\_killable}
  \ee
\item get --- gets the specified property
  \be
\item {\tt get\_early\_free}
\item {\tt get\_memo\_len}
\item {\tt get\_killable}
  \ee
\item set --- sets the specified property
  \be
\item {\tt set\_early\_free}
\item {\tt set\_memo\_len}
\item {\tt set\_killable}

  TODO: If we set \(x\) as killable and \(x\) has an nn vector \(y\), does \(y\)
  also become killable? \TBC
  
  \ee
\ee

It is always possible to perform a {\tt get} action. However, {\tt set} actions
and {\tt do} actions are limited. 

A set action can only be performed if 
\be
\item \(n_E = 0\)
\item \(e = \) false. Once a vector has been fully materialized, 
  these properties don't make sense.
\item Since we cannot have empty vectors, from the above, we can assert that set
  actions cannot be performed on fully materialized vectors where 
  \(e = \mathrm{true}\)
\item it has not already been performed e.g., once \(x:set\_memo(n)\) 
  has been performed, we cannot do \(x:set\_memo(m)\)
\ee

A do action can only be performed if 
\be
\item \(e = \mathrm{false}\) --- the vector has not been fully materialized
\item the appropriate {\tt set} action was performed earlier. By this we mean
  that
  \be
\item {\tt kill} action ignore if \(k = \mathrm{false}\)
\item {\tt early\_free} action ignore if \(f = \mathrm{false}\)
  \ee
  \ee

\end{document}

% TODO COnsider lma_to_chunks() and other lma interactions with above
\end{document}
