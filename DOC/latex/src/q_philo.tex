\startreport{Q --- An Analytics Framework}
\reportauthor{Ramesh Subramonian}

\newcommand{\YES}{\checkmark}


\section{Introduction}

Q is a vector language, designed for efficient counting implementation
of counting, sorting and data transformations. It uses a single data
structure --- a table.

\subsection{Motivation}

I will motivate the need for Q by quoting from two of my Gods --- Codd and
Iverson. I could be accused of quoting scripture for my purpose (see
below) and it is true that I am being selective in my extracts.
However, that does not detract from their essential verity.

\begin{verse}
The devil can cite Scripture for his purpose. \\ 
An evil soul producing holy witness \\ 
Is like a villain with a smiling cheek, \\
A goodly apple rotten at the heart:
\end{verse}

\subsubsection{Extracts from Codd}

{\tt 
The most important motivation for the research work that resulted in the
relational model was the objective of providing a sharp and clear
boundary between the logical and physical aspects of database
management. We call this the {\em data independence objective}.

A second objective was to make the model structurally simple, so that
all kinds of users and programmers could have a common understanding of
the data, and could therefore communicate with one another about the
database. We call this the {\em communicability objective}.

A third objective as to introduce high level language concepts *but not
specific syntax) to enable users to express operations upon large chunks
of information at a time. This entailed providing a foundation for
set-oriented processing (i.e., the ability to express in a single
statement the processing of multiple sets of records at a time). We
call this the {\em set-processing objective}.

To satisfy these three objectives, it was necessary to discard all those
data structuring concepts (e.g., repeating groups, linked structures)
that were not familiar to end users and to take a fresh look at
the addressing of data.
}

We have deviated from Codd's preference for the relational model.
Instead, we choose to drop down one level to the table. As Codd writes:

{\tt 
  Tables are at a lower level of abstraction than relations, since they
    give the impression that positional (array-type) addressing is
    applicable (which is not true of \(n\)-ary relations), and they fail
    to show that the information content of a table is independent of
    row order. Nevertheless, even with these minor flaws, tables are the
    most important conceptual representation of relations, because they
    are universally understood.

}

Lastly, in designing Q, we wanted it to be a data model as Codd defines
one

{\tt
A data model is a combination of at least three components:
  \be
  \item A collection of data structure types (the building blocks);
\item A collection of operators or rules of inference, which can be
  applied to any valid instances of the data types listed in (1), to
  retrieve, derive, or modify data from any parts of those structures in
  any combinations desired;
\item A collection of general  integrity rules, which implicitly or
  explicitly define the set of consistent database states or changes of
  states or both
  \ee

}

\subsubsection{Extracts from Iverson}

The importance of language has been stated over the centuries. Iverson
quotes the following from Whitehead:

{\tt By relieving the brain of all unnecessary work, a good notation
  sets it free to concentrate on more advanced problems, and in effect
    increases the mental power of the race}

In the same vein, he quotes Babbage:

{\tt The quantity of meaning compressed into small space by algebraic
  signs, is another circumstance that facilitates the reasonings we are
    accustomed to carry on by their aid}

I would hesitate to claim that Q meets any of the following criteria
that Iverson lays down for good notation. But it is  definitely the
guiding principle and aspirational goal for Q.

{\tt 
\be
\item Ease of expressing constructs arising in problems. If it is to be
effective as a tool of thought, a notation must allow convenient
expression not only of notions arising directly from a problem, but also
of those arising in subsequent analysis, generalization and
specialization.

\item Suggestivity. A notation will be said to be suggestive if the
forms of the expressions arising in one set of problems suggests related
expressions which find application in other problems.

\item Ability to subordinate detail. Brevity is achieved by
subordinating detail, and we will consider three important ways of doing
this
\bi
\item the use of arrays
\item the assignment of names to functions and variables
\item the use of operators
\ei

\item Economy. Economy requires that a large number of ideas be
expressible in terms of a relatively small vocabulary. 

\item Amenability to formal proofs

\ee
}




\subsection{Notations}
\begin{definition}
\(\delta(x, y)  = 1 \) if \(x = y\) and 0 otherwise
\end{definition}

\begin{notation}
For convenience, we shall often blur the distinction between arrays,
sets and tables.  Hence, \(T.f\) refers to field \(f\) of table
\(T\). Similarly, \(T[i].  f\) means the \(i^{th}\) value of field
\(f\) of table \(T\).  
\end{notation}

\begin{notation}
We shall often use \(T_X\) to refer to the table whose name is \(X\).
\end{notation}

\begin{notation}
Also, \(T|f(\ldots)\) refers to the subset of
\(T\) for which the predicate \(f(\ldots)\) is satisfied.
\end{notation}

\begin{notation}
\(Count(T)\) is cardinality of \(T\)
\end{notation}

\begin{notation}
\label{NumNN}
\(NumVal(T.f, v) = \sum_i \delta (T[i].f, v)\)
\end{notation}

\begin{notation}
\(Unique(T.f)\) be the unique values of \(T.f\)
\end{notation}

\begin{notation}
\(CU(T.f)\) is the count of the unique values of \(T.f =
Count(Unique(T.f))\)
\end{notation}

\begin{notation}
\((f_1:v_1 \ldots f_N:v_N) \in T \Rightarrow 
\exists j: T[j].f_1 = v_1 \wedge \ldots T[j].f_N = v_N \)
\end{notation}

\begin{notation}
\((v_1 \ldots v_N) = T[i].(f_1,\ldots f_N) \Rightarrow 
v_1 = T[i].f_1 \ldots v_N = T[j].f_N\)
\end{notation}

\begin{notation}
\(\exists ! \) is read as ``there exists a unique''
\end{notation}


\begin{definition}
Let \(T_1.f_1 = Join (T_2, f_2, l_2, l_1)\) be a field in \(T_1\)
  created by joining field \(f_2\) in Table \(T_2\) using \(l_2\) in
  \(T_2\) and \(l_1\) in \(T_1\) as the link fields.
\end{definition}


\section{System Capabilities}

\subsection{Field Types}
\label{Field_Types}
\be
\item wchar\_t Do we really support this all the way? \TBC 
\item char
\item bool
\item int
\item long long (LL)
\item float
\item double
\item char string. There are 2 variations
\be
\item sz = 0 means variable length
\item sz = \(L\) means constant length of \(L\) Or is this L+1 because
of null termination? \TBC
\ee
\ee

\be
\item If an operation is expected to create a table \(T\) and a table
by that name exists, then the original table is first deleted. Examples
of operations that create tables are
\be
\item add table --- Section~\ref{add_tbl}
\item load from CSV --- Section~\ref{dld}
\ee
\item 
If an operation is expected to create a field with name \(f\) and a
field with such a name exists, then the old field is deleted {\bf after}
the operation is performed and replaced by the newly created field. 
\item 
A table has a primary name (non-null) and a display name, which can be
null. The primary name must be unique. The display name, if non-null,
must be unique. 
\item Similary, fields have a primary name (non-null) and a display
name, which can be null. 
The primary name must be unique, within the table, in which the field
occurs. The display name, if non-null, must be unique, within the table.
\ee

%-----------------------------------------------------

\section{Functions}

\subsection{list\_tbls}
\label{list_tbls}

Returns list of tables in the docroot as a 3-column CSV text file
\be
\item  tbl ID
\item name
\item display name
\ee

%--------------------------------------------------------------
\subsection{list\_files}
\label{list_files}

Returns list of files in the docroot as a 2-column text file
\be
\item is\_external --- 1 means that it is owned by somebody else; else, 0
\item file\_name  --- unique, non-null string
\ee

%--------------------------------------------------------------

\subsection{add\_fld}
\label{add_fld}

Creates field \(f\) in table \(T\), whose values are stored in specified
data file.  Arguments are 
\be
\item \(T\) --- table
\item \(f\) --- field
\item attributes --- string. This is a colon separated concatenation of
\verb+name=value+ pairs where the names are
\be
\item \verb+sz+ --- size of field. Set to 0 for variable length fields
\item \verb+fldtype+ --- see Section~\ref{Field_Types}
\ee
\item \verb+nR+ --- number of rows
\item data file
\ee

Error conditions
\be
\item If table is empty, its number of rows is initially undefined. It
is set to \(n_R\) from input.  If table is non-empty, its number of rows
is defined, say, \(n_R'\). Then \(n_R = n_R'\)
\ee


\subsection{add\_tbl}
\label{add_tbl}
Arguments 
\be
\item \(T\) --- table
\item \(n_R\) --- number of rows. If not specified, set to 0.
\ee
Creates a table with name \(T\).

\subsection{is\_tbl}
\label{is_tbl}
Arguments 
\be
\item \(T\) --- table
\ee
If table \(T\) exists, output is \(1, tbl\_id)\). Else, 
output is \((0, -1)\)

{\bf Question?} Does the input argument refer to the primary name
or the display name or the ID? 
\bi
\item If the argument starts with {\tt display:}, then the text after 
this prefix is considered to be the display name.
\item If the argument starts with {\tt id:}, then the text after 
this prefix is considered to be the ID (must be a positive integer).
\item Else, the entire argument is considered to be the primary name.
\ei

\subsection{is\_fld}
\label{is_fld}
Arguments 
\be
\item \(T\) --- table
\item \(f\) --- field
\ee

If table \(T\) exists and field \(f\) exists in table \(T\), 
output is \(1, fld\_id)\). Else, output is \((0, -1)\)

{\bf Question?} Does the input argument (whether \(T\) or \(f\)) 
refer to the primary name or the display name or the ID? 
Same as in Section~\ref{is_tbl}.

\subsection{is\_nn\_fld}
\label{is_nn_fld}
Arguments 
\be
\item \(T\) --- table
\item \(f\) --- field
\ee

Returns 1 if \(f\) exists in \(T\) and it has a nn field. \(f\) must be
primary name.

\subsection{pr\_fld}
\label{pr_fld}
Arguments 
\be
\item \(T\) --- table
\item \(f\) --- field
\item \(f_c\) --- condition field or range. If null, all rows printed
\item \(O\) --- output file name. If null, writes to stdout
\ee

Prints all rows of field \(f\) of table \(T\). Behavior of print can be
modified by specifying third parameter as either
\be
\item \(f_c\) --- printing of \(T[i].f\) is suppressed if \(T[i].f_c = false\).
\item \(lb:ub\) --- we print rows with lb as lower bound inclusive and
ub as upper bound exclusive
\ee

If range is specified, following must be satisified
\be
\item \(lb > 0 \)
\item \(ub \leq n_R\)
\item \(lb < ub\)
\ee

\subsection{tbl\_meta}
\label{tbl_meta}

Arguments
\be
\item \(T\) --- table
\ee
Lists all fields (including auxiliary) in table \(T\). 
For each field, we print
\be
\item name
\item display name
\item field type --- Section~\ref{Field_Types}
\ee

\subsection{fld\_meta}
\label{fld_meta}

Arguments 
\be
\item \(T\) --- table
\item \(f\) --- field
\ee

For field \(f\) in table \(T\) (asuming it exists), we print the
following information as a 2-column CSV
\be
\item \verb+id+
\item \verb+tbl_id+
\item \verb+name+
\item \verb+disp_name+
\item \verb+filename+
\item \verb+fldtype+
\item \verb+auxtype+
\item \verb+n_sizeof+
\item \verb+parent_id+
\ee


\subsection{del\_tbl}
\label{del_tbl}

Arguments
\be
\item \(T\) --- table
\ee

Deletes table \(T\) and all fields in \(T\)

\subsection{del\_fld}
\label{del_fld}

Arguments 
\be
\item \(T\) --- table
\item \(f\) --- field
\ee

Deletes field \(f\) in table \(T\). If \(f\) is a primary field, then
all auxiliary fields (if any) are deleted as well. If \(is\_external = 0\), then the corresponding file is {\bf not} deleted; else, it is deleted.

\subsection{get\_nR}
\label{get_nR}
Arguments
\be
\item \(T\) --- table
\ee
Returns number of rows in table \(T\)

\subsection{set\_nR}
\label{set_nR}
Arguments
\be
\item \(T\) --- table
\item \(n_R\) --- positive integer
\ee

Sets number of rows in table \(T\) to \(n_R\). Number of rows in table
must be 0 beore operation.

\subsection{add\_aux\_fld}
\label{add_aux_fld}

\be
\item \verb+sz+ --- this indicates it is an auxiliary field
\item \verb+nn+ --- this indicates it is an auxiliary field
\item \verb+hash+ --- this indicates it is an auxiliary field
\ee

\subsection{dup\_fld}
\label{dup_fld}
Arguments are 
\be
\item \(T\) --- table
\item \(f_1\) --- field in \(T\)
\item \(f_2\) --- newly created field in \(T\)
\ee

Creates \(f_2\) in\(T\) which is a clone of \(f_1\). Requires
\be
\item \(f_1 \neq f_2\)
\ee

\subsection{fop}
\label{fop}
Arguments are 
\be
\item \(T\) --- table
\item \(f\) --- field in \(T\). Modified in situ
\item \(\circ\)  --- operation on \(f\) 
\ee

Operations supported are
\be
\item \verb+sortA+ --- \(\forall i: T[i].f \leq T[i+1].f\). 
\item \verb+sortD+ --- \(\forall i: T[i].f \geq T[i+1].f\)
\item \verb+lcase+ --- Applies {\tt tolower} on all ascii characters in 
string.  
\ee
Restrictions
\be
\item field cannot have null values
\ee
Operations supported are in Table~\ref{tbl_fop}
\begin{table}
\centering
\begin{tabular}{|l||l|l|l|l|l|l|l|}  \hline \hline
{\bf Operation} & {\bf bool} & {\bf char} & {\bf int} & {\bf LL}
& {\bf float } & {\bf double} & {\bf string} \\ \hline \hline
{\bf sortA} &   &   & \YES & \YES & \YES &   &   \\ \hline
{\bf sortD} &   &   & \YES & \YES & \YES &   &   \\ \hline
{\bf lcase} &   &   &   &   &   &   & \YES \\ \hline
\hline
\end{tabular}
\caption{Supported Operations for f\_to\_s}
\label{tbl_fop}
\end{table}

\subsection{f\_to\_s}
\label{ftos}
Produces a scalar from a field. Arguments are 
\be
\item \(T\) --- table
\item \(f\) --- field in \(T\). 
\item \(\circ\)  --- operation on \(f\) 
\ee

Operations supported are in Table~\ref{tbl_f_to_s}
\begin{table}
\centering
\begin{tabular}{|l||l|l|l|l|l|l|l|}  \hline \hline
{\bf Operation} & {\bf bool} & {\bf char} & {\bf int} & {\bf LL}
& {\bf float } & {\bf double} & {\bf string} \\ \hline \hline
{\bf sum} & \YES & \YES & \YES & \YES & \YES &   &   \\ \hline
{\bf min} &   &   & \YES & \YES & \YES &   &   \\ \hline
{\bf max} &   &   & \YES & \YES & \YES &   &   \\ \hline
{\bf avg} &   &   & \YES & \YES & \YES &   &   \\ \hline
\hline
\end{tabular}
\caption{Supported Operations for f\_to\_s}
\label{tbl_f_to_s}
\end{table}


\subsection{is\_a\_in\_b}
\label{is_a_in_b}

\be
\item \(T_1\) --- table
\item \(f_1\) --- field in \(T_1\). 
\item \(T_2\) --- table
\item \(f_2\) --- field in \(T_2\). 
\item \(f_c\) --- newly created field in \(T_1\). 
\item \({f_2}^S\) --- field in \(T_2\) used as source
\item \({f_1}^D\) --- newly created field in \(T_1\) 
\ee

The are two variations of this call
\be
\item If \(f_c\) is provided, then \({f_2}^S\) and \({f_1}^D\) must be
  null. In this case, 
\(\forall i: T[i].f_1 \in T_2.f_2 \Rightarrow T[i].f_c \leftarrow true\)
\item If \(f_c\) is {\bf not} provided, then \({f_2}^S\) and \({f_1}^D\) must be
  provided. In this case, 
  \TBC
\ee

\subsection{f1opf2}
\label{f1opf2}

Arguments are 
\be
\item \(T_1\) --- table
\item \(f_1\) --- field in \(T_1\)
\item \(\circ\)  --- operation to b performed on \(f_1\)
\item \(f_2\) --- newly created field in \(T_1\)
\ee

Creates field \(f_1\) in \(T_1\) by performing operation \(\circ\) on
\(f_1\) in \(T_1\). 
Operations supported are in Table~\ref{tbl_f1opf2}. Detailed
explanations in Section~\ref{details_f1opf2}.
\begin{table}
\centering
\begin{tabular}{|l||l|l|l|l|l|l|l|}  \hline \hline
{\bf Operation} & {\bf bool} & {\bf char} & {\bf int} & {\bf LL}
& {\bf float } & {\bf double} & {\bf string} \\ \hline \hline
{\bf conv    } & \YES &      & \YES & \YES & \YES & \YES &   \\ \hline
{\bf bitcount} &      &      & \YES & \YES &      &      &   \\ \hline
{\bf sqrt    } &      &      &      &      & \YES &      &   \\ \hline
\verb+!+       & \YES &      & \YES & \YES &      &      &   \\ \hline
\verb=++=      &      &      & \YES & \YES &      &      &   \\ \hline
\verb+--+      &      &      & \YES & \YES &      &      &   \\ \hline
\verb+~+       &      &      & \YES & \YES &      &      &   \\ \hline
{\bf hash}     &      &      &      &      &      &      & \YES \\ \hline
{\bf shift   } &      &      & \YES &      &      &      &   \\ \hline
{\bf cum     } &      &      & \YES & \YES &      &      &   \\ \hline
\hline
\end{tabular}
\caption{Supported Operations for f1opf2}
\label{tbl_f1opf2}
\end{table}

\subsubsection{Details for f1opf2}
\label{details_f1opf2}
We explain each of the operations below
\bd
\item [CONV] Converts from one type to another. Example
\verb+op=conv:newtype=long long+. Conversions supported
are
\be
\item LL \(\rightarrow\) int
\item LL \(\rightarrow\) bool
\item LL \(\rightarrow\) float
\item int \(\rightarrow\) bool
\item int \(\rightarrow\) LL
\item int \(\rightarrow\) float
\item bool \(\rightarrow\) int
\item bool \(\rightarrow\) LL
\item float \(\rightarrow\) int
\item float \(\rightarrow\) LL
\item float \(\rightarrow\) double
\item double \(\rightarrow\) int
\item double \(\rightarrow\) LL
\item double \(\rightarrow\) float
\ee
\item [BITCOUNT] 
Counts the number of bits.
Example \verb+op=bitcount+
\(Type(f_2) = int\)
\item [SQRT] 
Operation is \(\sqrt{x}\). 
Example \verb+op=sqrt+
\(Type(f_2) = Type(f_1)\)
\item [NEGATION] Toggles bits. 
Example is \verb+op=!+
\(Type(f_2) = Type(f_1)\)
\item [ONE's COMPLEMENT] 
Takes one's complement. 
Example is \verb+op=~+
\(Type(f_2) = Type(f_1)\)
\item [INCREMENT]
Increments by 1
Example is \verb~op=++~
\(Type(f_2) = Type(f_1)\)
\item [DECREMENT]
Decrements by 1
Example is \verb~op=--~
\(Type(f_2) = Type(f_1)\)
\item [CUM] 
Accumulates the values of \(f_1\). Basic version is as follows.
\(i = 0 \Rightarrow f_2[i] = f_1[i]\); else, \(f_2[i] = f_1[i] + f_2[i-1]\)
\(Type(f_2) = Type(f_1)\)

A subtlety arises when invoked as follows
\verb+op=cum:reset_on=N:reset_to=M+ The way to think about this is to
assume sequential ordered execution for ascending values of \(i\) and
then \(f_1[i] = N \Rightarrow f_2[i] \leftarrow M\)

\item [SHIFT] Shifts the values up or down up to a maximum of 16
Example is \verb+op=shift:val=N+. In this case,
\(f_2[i] = f_1[i+N]\). If \(N > 0 \), then it like a shift down
of the column i.e., the first \(N\) values of \(f_2\) are undefined. If
\(N < 0\), then it is like a shift up and the last \(N\) values will be
undefined. Note that \(N \neq 0\). Currently, \(|N| \leq 16\).

\item [HASH] Creates a has of the string. A few options to note
\be
\item One must specify whether to create a 4-byte hash or an 8-byte hash
with the specifier \verb+len=4+ or \verb+len=8+
\item The default algorithm is an MD5 hash. One can specify an alternate
alorithm with the specifier \verb+algo=addchars+. In this case, we
compute the ``hash'' by treating each character as a number whose maximum
value is 256 and adding up these numbers, one at a time, shifting the
accumulated value by 8 bits. Consider the string {\tt ABC} where {\tt A}
has value 65. In this case, the ``hash'' would be \(65 \times {256}^2 + 66
\times 256 + 67\). Useful to create a unique integer for short-length
strings.
\ee
\ed

\subsection{crossprod}
\label{crossprod}

Arguments are 
\be
\item \(T_1\)
\item \(f_1\)
\item \(T_2\)
\item \(f_2\)
\item \(T_3\)
\ee

Creates a new table \(T_3\) with fields \(f_1\) and \(f_2\).  If such a
table exists, it is deleted. 
Let \(n_1\) be number of non-null values of \(f_1\) in \(T_1\).
Let \(n_2\) be number of non-null values of \(f_2\) in \(T_2\).
Then, \(T_3\) will have \(n_1 \times n_2\) rows. 
For each row of \(T_1\) which has a non-null value of \(f_1\)
for each row of \(T_2\) which has a non-null value of \(f_2\),
we create a row of \(T_3\) with values \(T_1[i].f_1, T_2[j].f_2\). Note
that duplicate values in \(f_1\) or \(f_2\) will result in duplicate
values in \(T_3\). It is the responsibility of the caller to create
unique values if so desired.

\subsection{join}
\label{join}

\be
\item \(T_s\) --- table
\item \(l_s\) --- link field in \(T_1\). 
\item \(f_s\) --- source field in \(T_1\). 
\item \(T_d\) --- table
\item \(l_d\) --- link field in \(T_2\). 
\item \(f_d\) --- newly created field in \(T_2\). 
\ee

Let \(T_d[i].l_d = v_d\). \(T_d[i].f_d \la T_s[l_s = v_d].f_s\). Note
that
\be
\item \(|T_s[l_s = v_d].f_s| = 0 \Rightarrow T_d[i] = \bot\)
\item \(|T_s[l_s = v_d].f_s| > 1 \Rightarrow T_d[i] = \) is arbitrarily
set to some one value of \(T_s[l_s = v_d].f_s\)
\ee

\subsection{wisifxthenyelsez}
\label{wisifxthenyelsez}

\be
\item \(T\) --- table
\item \(w\) --- output field 
\item \(x\) --- boolean field 
\item \(y\) --- 
\item \(z\) --- 
\ee

Creates field \(w\), whose type is same as that of field \(y\).
Algorithm in Figure~\ref{pseudo_code_wisifxthenyelsez}. Restrictions
\be
\item field types for \(y\) and \(z\) must be the same
\item field types for \(y\) must be one of bool, integer, long, float
\ee

\begin{figure}
\centering
\begin{tabbing} \hspace*{0.25in} \=  \hspace*{0.25in} \= \kill
{\bf if} \(T_x[i] = \bot\) then \+ \\
  \(T_w[i] \la \bot \) \- \\
{\bf else} \+  \\
  {\bf if} \(T_x[i] = true\) then \+  \\
    \(T_w[i] \la T_y[i] \) \- \\
  {\bf else} \+  \\
    \(T_w[i] \la T_z[i] \) \- \\
  {\bf endif} \-  \\
{\bf endif} 
\end{tabbing}
\label{pseudo_code_wisifxthenyelsez}
\caption{Pseudo-code for wisifxthenyelsez}
\end{figure}


\subsection{count\_vals}
\label{count_vals}

Arguments are 
\be
\item \(T_1\) --- input table 
\item \(f_1\) --- input field  in \(T_1\)
\item \(f^c_1\) --- optional counter field in \(T_1\)
\item \(T_2\) --- output table 
\item \(f_2\) --- output field in \(T_1\)
\item \(f^c_2\) --- counter field in \(T_2\)
\ee

Field types supported are in Table~\ref{tbl_fldtypes_count_vals}. Other
restrictions are
\be
\item \(fldtype(f_1) = fldtype(f^c_1)\)
\item \(fldtype(f_2) = fldtype(f^c_2)\)
\ee

We create a table \(T_2\) where \(f_2\) is the unique, non-null values
of \(f_1\) in \(T_1\). If \(f^c_1\) is not specified, \(f^c_2[i] =
|T_1[f_1 = T_2[f^c-2[i]]|\), the number of times the corresponding
value of \(f_1\) occurred in \(T_1\). If it is specified, then the
output is computed in much the same way but weighted by \(f^c_1\). In
other words, let \(T_2[j].f_2 = V\). Then, 
\begin{displaymath}
T_2[j].f^c_2 = \sum_i T_1[i].f^c_1[i] \times \delta(T_1[i].f_1, V) 
\end{displaymath}

\begin{table}
\centering
\begin{tabular}{|l||l|l|l|l|l|l|l|}  \hline \hline
{\bf Field} & {\bf bool} & {\bf char} & {\bf int} & {\bf LL}
& {\bf float } & {\bf double} & {\bf string} \\ \hline \hline
\(f_1\) &   &   & \YES & \YES &      &   &   \\ \hline
\(f^c_1\) &   &   & \YES & \YES &      &   &   \\ \hline
\(f_2\) &   &   & \YES & \YES &      &   &   \\ \hline
\(f^c_2\) &   &   & \YES & \YES &      &   &   \\ \hline
\hline
\end{tabular}
\caption{Supported Field Types for count\_vals}
\label{tbl_fldtypes_count_vals}
\end{table}

\sunbsection{Enhancements}
Do not create an output count field if not desired. % TODO

\subsection{bindmp}
\label{bindmp}
Arguments are
\be
\item \(T\) --- input table
\item \(f_1:f_2:\ldots f_N\) --- list of fields separated by colon. Must
have at least one field
\item \(f_c\) --- boolean field in \(T_1\) optional
\item \(F\) --- output file to be created
\ee

Creates file \(F\) by dumping the desired fields, a row at a time. A
row is output if either \(f_c\) is not specified or \(f_c\) is specified
and it is true. 

Restrictions are
\be
\item All fields do not need to have the same type 
\item Field type must be one of 
\be
\item char (C)
\item bool (B)
\item int (I)
\item long long (L)
\item float (F)
\item double (D)
\ee
\ee

\subsection{binld}
\label{binld}
Similar to bindmp in Section~\ref{bindmp}.
Arguments are
\be
\item \(T\) --- table to be created
\item \(f_1:f_2:\ldots f_N\) --- list of fields separated by colon. Must
have at least one field
\item \(t_1:t_2:\ldots t_N\) --- where \(t_i\) is specifier of field
type for field \(f_i\). 
\item \(F\) --- input file to be created
\ee

Creates table \(T\) from file \(F\). Restrictions are 
\be
\item \(t_i \in \{B, C, I, LL, F, D\}\)
\ee

\subsection{copy\_fld}
\label{copy_fld}

Creates field \(f_2\) table \(T_2\) by copying
\(T[i].f\) to \(T_2\) if \(T_1[i].f_c = true\). The order is preserved
in the copy operation.
Arguments are
\be
\item \(T_1\) --- input table 
\item \(f_1\) --- input field 
\item \(f_c\) --- optional condition field 
\item \(T_2\) --- output table 
\item \(f_2\) --- newly created output field 
\ee
Restrictions are
\be
\item \(T_2\) must exist
\item If \(f_c\) not specified, \(|T_1| = |T_2|\). Else,
  \(NumVal(T_1.f_c, true), = |T_2|\)
\item Field type must be one of 
\be
\item char
\item bool
\item int
\item long long (LL)
\item float
\item double
\ee
\ee

\subsection{sortf1f2}
\label{sortf1f2}

Arguments are
\be
\item \(T_1\) --- input table 
\item \(f_1\) --- input field 
\item \(f_2\) --- optional input field 
\item \(sort~spec\) --- see below
\ee
Restrictions are
\be
\item \(f_1 \neq f_2\)
\item Field type must be one of 
\be
\item int
\item long long (LL)
\ee
\item sort spec must be one of 
\be
\item \verb+A+  --- sort \(f_1\) ascending. \( \Rightarrow f_2 = \bot\)
\item \verb+D+  --- sort \(f_1\) descending. \( \Rightarrow f_2 = \bot\)
\item \verb+AA+ --- \(f_1\) primary ascending and \(f_2\) secondary ascending
\item \verb+A_+ --- \(f_1\) primary ascending and \(f_2\) drag along
\item \verb+AD+ --- \(f_1\) primary ascending and \(f_2\) secondary descending
\item \verb+DA+ --- \(f_1\) primary descending and \(f_2\) secondary ascending
\item \verb+D_+ --- \(f_1\) primary descending and \(f_2\) drag along
\item \verb+DD+ --- \(f_1\) primary descending and \(f_2\) secondary descending
\ee
\ee

\subsection{import}
\label{import}
Arguments are
\be
\item \(D_{from}\) --- from docroot 
\item \(T_{from}\)
\item \(T_{to}\)
\ee
Copies the table \(T_{from}\) from the docroot to the current docroot,
       \((D_{to}\)
Notes
\be
\item \(D_{from} \neq D_{to}\)
\item If \(T_{to}\) exists in \(D_{to}\), then it is deleted
\item If we delete a field in \(D_{to}\) or the entire table itself,
there is no impact on \(T_{from}\) in \(D_{from}\). (As an
  implementation note, this is because we mark the imported fields with
  an \verb+is_external+ flag and we do not allow modifications of these
  files. This is specially important for operations which modify the
  underyling storage like Section~\ref{fop}.
\ee

\subsection{rename}
\label{rename}
Two styles of invocation. 
\be
\item to rename a table 
\be
\item \(T_1\) --- old name of table
\item \(T_2\) --- new name. If such a table exists, it is deleted
\ee
\item to rename a field 
\be
\item \(T_1\) --- name of table
\item \(f_1\) --- old name of field 
\item \(f_2\) --- new name of field. If such a field exists, it is
deleted
\ee
\ee

\subsection{drop\_null\_fld}
\label{drop_null_fld}
Arguments are 
\be
\item \(T\) --- table
\item \(f\) --- field
\ee
If \(T.f\) has an auxiliary nn field, it is deleted i.e., all values of
\(f\) are now defined. It is important to be sure that you know the
value of the field in its undefined state. Usually, this will be 0, but
you had better be damn sure!

\subsection{no\_op}
\label{no_op}

As expected, does nothing. Although it writes a line to the log file. 

\subsection{dld}
\label{dld}

\TBC
