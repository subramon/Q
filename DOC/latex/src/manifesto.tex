\documentclass[12pt,letterpaper]{article}
\usepackage{times}
\usepackage{helvet}
\usepackage{courier}
\usepackage{hyperref}
\usepackage{fancyheadings}
\pagestyle{fancy}
\usepackage{pmc}
\usepackage{graphicx}
\setlength\textwidth{6.5in}
\setlength\textheight{9.0in}
\begin{document}
\title{The case for ``Medium Data''}
\author{ Ramesh Subramonian }
\maketitle
\thispagestyle{fancy}
\lhead{}
\chead{}
\rhead{}
\lfoot{}
\cfoot{Ramesh Subramonian}
\rfoot{{\small \thepage}}

Marc Andreesen has claimed that ``software is eating the world''. It is more
likely to burn up the world before it has a chance to finish.

The introduction of map-reduce \cite{mapreduce2004} both legitimized and spawned a plethora
of distributed solutions to data and compute intensive problems. Unfortunately,
as McSherry has noted --- ``Big data systems may scale well, but this can often
be just because they introduce a lot of overhead.''

This phenomenon has not gone unnoticed. 
McSherry asks us to account for the COST of scalability in \cite{McSherry2015}.
From the Kx systems blog, ``The Semiconductor Industry Association (SIA)
recently reported that in roughly 20 years today’s computer chips will require
more power than global energy production can provide.''
(\url{https://kx.com/blog/kdb-enables-green-computing/})

Fuel efficiency standards for cars were introduced in 1975. LEED standards for
buildings were
established in 1994. It is unthinkable to believe that software should not be
subject to similar standards.

The response of cloud compute providers to the very real pain felt by CIOs is
not dissimlar from that of pharmaceutical companies. 
Computation is headed down its own opiod crisis without an intervention.


The need for scale-out is not debatable --- if your problem cannot be solved on
one machine in a reasonable time frame, you need to distribute the workload.
However, that does not obviate the need to squeeze every last bit out of each
machine. 


While we are likely to find supporters of ``green computing'' (
\cite{Limits2018}), our alternatives must provide significant cost-savings. 
My experiences at LinkedIn and Target has convinced me that this is possible.
While ``big data'' may all be the rage, the truth is that most problems are
better characterized as ``medium data''. We define {\em medium data} to be
problems that are big enough to require performance engineering yet small enough
to fit on one well-provisioned server.

It is human nature to suffer from ``Google-envy'' which drives one to adopt
techniques that while entirely appropriate for the top end are wholly mismatched to one's
actual needs. But envy is quickly coming up against the hard stop of the cost
incurred from indiscriminate computing.

Thesis: {\em I believe there is a market opportunity in providing highly 
performant replacements for many current compute hogs. }
\bibliographystyle{alpha}
\bibliography{../../Q_PAPER/ref}
\end{document}
