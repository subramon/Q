\documentclass[letterpaper,12pt]{article}
\usepackage{helvet}
\usepackage{hyperref}
\usepackage{fancyheadings}
\pagestyle{fancy}
\usepackage{graphicx}
\setlength\textwidth{6.5in}
\setlength\textheight{8.5in}
\newtheorem{problem_statement}{Problem Statement}
\newcommand{\TBC}{\framebox{\textbf{TO BE COMPLETED}}}
\newtheorem{assumption}{Assumption}
\newcommand{\be}{\begin{enumerate}}
\newcommand{\ee}{\end{enumerate}}
\newcommand{\bi}{\begin{itemize}}
\newcommand{\ei}{\end{itemize}}
\newcommand{\bd}{\begin{description}}
\newcommand{\ed}{\end{description}}
\newtheorem{notation}{Notation}
\begin{document}
\title{Multi-threading in qjit}
\author{Ramesh Subramonian}
\maketitle
\thispagestyle{fancy}
\lhead{}
\chead{}
\rhead{}
% \lfoot{{\small Decision Sciences Team}}
\cfoot{}
\rfoot{{\small \thepage}}

 \begin{abstract}
\ Q can be run in a collaborative mode, where it listens for commands over
 HTTP. This document describes the design and usage of this capability.
\end{abstract}

\section{Configurations}

\be
\item To enable the web server
  \be 
\item Set \verb+is_webserver = true+  Default is false
\item Set \verb+web_port+
  \ee
\item To enable the out of band server, which allows one to 
  make changes to configurations after {\tt Q} has started.
  \be
\item 
  Set \verb+is_out_of_band = true+ Default is false
\item Set \verb+out_of_band_port+ 
  \ee
  Note that the out of band server can be accessed {\bf only} over the  internal loopback
  interface.
  It cannot be accessed from a machine other than the one where {\tt Q} is
  running.
\item Set \verb+initial_master_interested+ to false if you want to run in a
  headless manner (Section~\ref{headless}. 
  

\section{Headless Server}

To run in headless mode, do something like: (explained in
\url{https://www.lua.org/pil/1.1.html})
\begin{verbatim}
qjit -i -lfoo -lbar
\end{verbatim}

After \tt{Q} executes {\tt foo.lua} and {\tt bar.lua}, the master thread 
enters an endless loop. In each iteration, all it does is sleep for one  second.

If you want to wake the master from its beauty sleep, you need to use the out
of band server and execute \url{/SetMaster?Status=On}. To put it back to sleep,  
execute \url{/SetMaster?Status=Off}. 

\ee
\end{document}


