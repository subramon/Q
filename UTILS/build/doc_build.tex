\startreport{Documenting the Build Process}
\reportauthor{Ramesh Subramonian}

\section{Relevant Environment Variables}
\be
\item Q\_BUILD\_DIR e.g., \verb+/tmp/q/+
\item Q\_SRC\_ROOT e.g., \verb+/home/subramon/Q/+
\item Q\_ROOT e.g., \verb+/home/subramon/local/Q/+
\ee

\subsection{Derived Environment Variables}
\be
\item CDIR e.g., \verb+/tmp/q/src/+ (from Q\_BUILD\_DIR)
\item HDIR e.g., \verb+/tmp/q/include/+ (from Q\_BUILD\_DIR)
\ee

\section{struct files}
\label{struct_files}

This is an important file. If you have {\tt typedef struct} in your .h file, it
needs to be listed here. This is because we put together a bunch of .h files
into a single {\tt q\_core.h} file and then cdef that. The order in which the individual {\tt .h} files are concatenated is important.

\section{copy generated files}
\label{copy_generated_files}

The aim of {\tt copy\_generated\_files.lua} is to get all the auto-generated 
{\tt .c} files into one directory.
and all {\tt .h} files into another directory. It calls {\tt
recursive\_copy.lua}. It is called by the Makefile

\bi
\item Starting from a root directory (Q\_SRC\_ROOT), 
\item traverse sub-diretoriies recursively, 
\item ignoring sub-directories that do {\bf NOT} match a particular
pattern (e.g., \verb+gen_src+ or \verb+gen_inc+), 
\item copying all files that exist in
those directories to a specified destination directory (e.g., \verb+/tmp/q/src/+
or /verb+/tmp/q/include+)
\ei


What about non-auto-generated files? \TBC

\section{Make so}

The script {\tt mk\_so.lua} is called by the Makefile and calls
\be 
\item {\tt chk\_env\_vars.lua}
\item {\tt is\_struct\_file.lua}
\item {\tt add\_h\_files\_to\_list.lua}
\ee

\subsection{Pre-requisites}
It assumes that 
\be
\item all necessary .c files have been copied into CDIR
\item all necessary .h files have been copied into HDIR
\ee

\subsection{What it does}

The script \verb+mk_so.lua+ gets called twice, 
once to build q\_basic and once to build q\_core.
In both cases, it needs to create 2 things, both of which are needed so that we
can invoke C functionality from LuaJIT.
\be
\item \verb+q_core.h+ -- this is given to {\tt ffi.cdef}
\item \verb+libq_core.so+ -- this is given to {\tt ffi.load}
\ee
These are copied into the target locations e.g.,
\be
\item \verb+/home/subramon/local/Q/include/q_core.h+
\item \verb+/home/subramon/local/Q/lib/libq_core.so+
\ee

\section{Makefile}

\subsection{q core}
\label{q_core}

These represent the basic functions that must be compiled.

\section{Appendix}

Bogus reference: \cite{sarawagi99}
\bibliographystyle{alpha}
\bibliography{../../DOC/Q_PAPER/ref} 




