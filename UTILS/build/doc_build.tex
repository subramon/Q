\startreport{Documenting the Build Process}
\reportauthor{Ramesh Subramonian}

\section{Relevant Environment Variables}
\be
\item Q\_BUILD\_DIR e.g., \verb/tmp/q/+
\item Q\_SRC\_ROOT e.g., \verb+/home/subramon/WORK/Q/+
\item Q\__ROOT e.g., \verb+/home/subramon/local/Q/+
\ee

\subsection{Derived Environment Variables}
\be
\item CDIR --- Q\_BUILD\_DIR + \/src/ e.g., \verb+/tmp/q/src/+
\item HDIR --- Q\_BUILD\_DIR + \/include/ e.g., \verb+/tmp/q/include/+
\ee

\section{copy generated files}
\label{copy_generated_files}

Starting from a root directory (Q\_SRC\_ROOT), traverse sub-diretoriies
recursively, ignoring sub-directories that do {\bf NOT} match a particular
pattern (e.g., \verb+gen_src+ or \verb+gen_inc+), copy all files that exist in
those directories to a specified destination directory (e.g., \verb+/tmp/q/src/+
or /verb+/tmp/q/include+)

The aim of this is to get all the auto-generated files into one directory.

What about non-auto-generated files? \TBC

\begin{table}[hbtp]
\centering
\begin{tabular}{|l|l|l|} \hline \hline
                & {\bf Calls} & {\bf Called by} \\ \hline \hline
copy\_generated\_files & recursive\_copy & \\ \hline
recursive\_descent &  exclude\_file, exclude\_dir & \\ \hline
\hline
\end{tabular}
\caption{Call graph}
\label{tbl_call_graph}
\end{table}

\section{Makefile}

\subsection{q core}
\label{q_core}

These represent the basic functions that must be compiled.

\section{Appendix}

Bogus reference: \cite{sarawagi99}
\bibliographystyle{alpha}
\bibliography{../../DOC/Q_PAPER/ref} 




