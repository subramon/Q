% \usepackage{hyperref}
\startreport{Creating Input to Aggregator}
\reportauthor{Ramesh Subramonian}

\section{Introduction}

\subsection{Related Work}
Relevant prior work can be found in \cite{sarawagi99}
and \cite{Fagin05}
\section{Raw Input}
Following \cite{Fagin05}, we assume that the input is a 
single relational table where columns are either dimensions or measures. 

Dimensions are
attributes along which aggregations will be performed e.g., time, age,
gender, product category. These may be numeric, like time or
categorical, like gender.
Example of dimension columns: time of sale, store ID, product category \ldots

Measures are numerical and are the target of aggregations. Examples are sale
price, number of items, \ldots We might want to know how average sale price
differs based on Gender. Here, Price and Number are the measures and Gender is
the dimension.

In our context, Each row refers to customer transaction. It could alternatively
refer to a document. 

We distinguish between raw columns and derived columns. As an example, see
Table~\ref{derived_columns}. In this case, 
\be
\item the raw attribute TimeOfSale
has 4 derived attributes, DayOfWeek, Month, Quarter, IsWeekend
\item the raw attribute StoreId
has 3 derived attributes, District, State, RegionalDistributionCenter
\item the raw attribute Age
2as 3 derived attributes, Generation and AgeBand
\item 
\ee

\begin{table}[hbtp]
\centering
\begin{tabular}{|l|l|l|l|l|l|l|l|l|l|l|} \hline \hline
Raw Column & Derived 1 & Derived 1 & Derived 3 & Column 4 \\ \hline
\hline
TimeOfSale & DayOfWeek & Month & Quarter & IsWeekend \\ \hline
StoreId     & District & State & RegionalDistributionCenter & --- \\ \hline
Age          & Generation & AgeBand & --- & --- \\ \hline
\end{tabular}
\caption{Examples of derived columns}
\label{derived_columns}
\end{table}
%-------------------------------------------------------

Discuss how raw input is converted to input to aggregator. \TBC

\section{Input to Aggregator}

We assume that the input is provided as follows. Let there be 3 raw attributes, \(a,
b, c\). Assume that 
\be
\item \(a\) has 2 derived attributes \(a_1, a_2\)
\item \(b\) has 3 derived attributes \(b_1, b_2, b_3\)
\item \(c\) has 4 derived attributes \(c_1, c_2, c_3, c_4\)
\ee
We will introduce a dummy derived attribute, \(a_0, b_0, c_0\), for each raw
attribute. The value of this attribute is always 0.
The input looks like Table~\ref{example_input_to_agg}.

\begin{table}[hbtp]
\centering
\begin{tabular}{|l|l|l|l|l|l|l|l|l|l|l|} \hline \hline
{\bf Derived Column} & {\bf Value} \\ \hline \hline
\(a_0\) & 0 \\ \hline
\(a_1\) & \(v_{a, 1}\) \\ \hline
\(a_2\) & \(v_{a, 2}\) \\ \hline
%
\(b_0\) & 0 \\ \hline
\(b_1\) & \(v_{b, 1}\) \\ \hline
\(b_2\) & \(v_{b, 2}\) \\ \hline
\(b_3\) & \(v_{b, 3}\) \\ \hline
%
\(c_0\) & 0 \\ \hline
\(c_1\) & \(v_{c, 1}\) \\ \hline
\(c_2\) & \(v_{c, 2}\) \\ \hline
\(c_3\) & \(v_{c, 3}\) \\ \hline
\(c_4\) & \(v_{c, 4}\) \\ \hline
%
\end{tabular}
\caption{Result of processing one row of original input}
\label{example_input_to_agg}
\end{table}
%-------------------------------------------------------

Let \(N(x) \) be the number of derived attributes for \(x\).
Hence, in our example, \(N(a) = 2+1 = 3\), \(N(b) = 3+1 = 4\), \(N(c) = 4+1 = 5\). 
This means that consuming one row of the input will cause \(\prod_x N(x)\) writes to the
aggregator. In our example, this means \((3 \times 4 \times 5) -1 = 59\). 
Using our current example, the keys generated 
are shown in Table~\ref{example_keys_to_aggregator}. We haven't shown all rows
but we hope the reader gets the picture. 

\newpage
{\small 
\begin{table}[hbtp]
\centering
\begin{tabular}{|c|l|l|l|l|l|l|l|l|l|l|l|l|l|l|l|} \hline \hline
{\bf ID} & {\bf Key} & \(a_0\) & \(a_1\) & \(a_2\) & 
\(b_0\) & \(b_1\) & \(b_2\) & \(b_3\) &
\(c_0\) & \(c_1\) & \(c_2\) & \(c_3\) & \(c_4\) \\ \hline \hline
 %--------------------
 01 & \(a_0 = \bot, b_0 = \bot, c_0 = \bot\)      & X & & & X & & & & X & & & & \\ \hline
 02 & \(a_0 = \bot, b_0 = \bot, c_1 = v_{c, 1}\)  & X & & & X & & & & & X & & & \\ \hline
 03 & \(a_0 = \bot, b_0 = \bot, c_2 = v_{c, 2}\)  & X & & & X & & & & & & X & & \\ \hline
 04 & \(a_0 = \bot, b_0 = \bot, c_3 = v_{c, 3}\)  & X & & & X & & & & & & & X & \\ \hline
 05 & \(a_0 = \bot, b_0 = \bot, c_4 = v_{c, 4}\)  & X & & & X & & & & & & & & X \\ \hline
 %--------------------
 06 & \(a_0 = \bot, b_1 = v_{b, 1}, c_0 = \bot\)      & X & & & & X & & & X & & & & \\ \hline
 07 & \(a_0 = \bot, b_1 = v_{b, 1}, c_1 = v_{c, 1}\)  & X & & & & X & & & & X & & & \\ \hline
 08 & \(a_0 = \bot, b_1 = v_{b, 1}, c_2 = v_{c, 2}\)  & X & & & & X & & & & & X & & \\ \hline
 09 & \(a_0 = \bot, b_1 = v_{b, 1}, c_3 = v_{c, 3}\)  & X & & & & X & & & & & & X & \\ \hline
 10 & \(a_0 = \bot, b_1 = v_{b, 1}, c_4 = v_{c, 4}\)  & X & & & & X & & & & & & & X \\ \hline
 %--------------------
 11 & \(a_0 = \bot, b_2 = v_{b, 2}, c_0 = \bot\)      & X & & & & & X & & X & & & & \\ \hline
 12 & \(a_0 = \bot, b_2 = v_{b, 2}, c_1 = v_{c, 1}\)  & X & & & & & X & & & X & & & \\ \hline
 13 & \(a_0 = \bot, b_2 = v_{b, 2}, c_2 = v_{c, 2}\)  & X & & & & & X & & & & X & & \\ \hline
 14 & \(a_0 = \bot, b_2 = v_{b, 2}, c_3 = v_{c, 3}\)  & X & & & & & X & & & & & X & \\ \hline
 15 & \(a_0 = \bot, b_2 = v_{b, 2}, c_4 = v_{c, 4}\)  & X & & & & & X & & & & & & X \\ \hline
%--------------------
 16 & \(a_0 = \bot, b_3 = v_{b, 3}, c_0 = \bot\)      & X & & & & & & X & X & & & & \\ \hline
 17 & \(a_0 = \bot, b_3 = v_{b, 3}, c_1 = v_{c, 1}\)  & X & & & & & & X & & X & & & \\ \hline
 18 & \(a_0 = \bot, b_3 = v_{b, 3}, c_2 = v_{c, 2}\)  & X & & & & & & X & & & X & & \\ \hline
 19 & \(a_0 = \bot, b_3 = v_{b, 3}, c_3 = v_{c, 3}\)  & X & & & & & & X & & & & X & \\ \hline
 20 & \(a_0 = \bot, b_3 = v_{b, 3}, c_4 = v_{c, 4}\)  & X & & & & & & X & & & & & X \\ \hline
%--------------------
%--------------------
 21 & \(a_1 = v_{a, 1}, b_0 = \bot, c_0 = \bot\)      & & X & & X & & & & X & & & & \\ \hline
 22 & \(a_1 = v_{a, 1}, b_0 = \bot, c_1 = v_{c, 1}\)  & & X & & X & & & & & X & & & \\ \hline
 23 & \(a_1 = v_{a, 1}, b_0 = \bot, c_2 = v_{c, 2}\)  & & X & & X & & & & & & X & & \\ \hline
 24 & \(a_1 = v_{a, 1}, b_0 = \bot, c_3 = v_{c, 3}\)  & & X & & X & & & & & & & X & \\ \hline
 25 & \(a_1 = v_{a, 1}, b_0 = \bot, c_4 = v_{c, 4}\)  & & X & & X & & & & & & & & X \\ \hline
%-------------------
 26 & \(a_1 = v_{a, 1}, b_1 = v_{b, 1}, c_0 = \bot\)      & & X & & & X & & & X & & & & \\ \hline
 27 & \(a_1 = v_{a, 1}, b_1 = v_{b, 1}, c_1 = v_{c, 1}\)  & & X & & & X & & & & X & & & \\ \hline
 28 & \(a_1 = v_{a, 1}, b_1 = v_{b, 1}, c_2 = v_{c, 2}\)  & & X & & & X & & & & & X & & \\ \hline
 29 & \(a_1 = v_{a, 1}, b_1 = v_{b, 1}, c_3 = v_{c, 3}\)  & & X & & & X & & & & & & X & \\ \hline
 30 & \(a_1 = v_{a, 1}, b_1 = v_{b, 1}, c_4 = v_{c, 4}\)  & & X & & & X & & & & & & & X \\ \hline
%-----------------
 31 & \(a_1 = v_{a, 1}, b_2 = v_{b, 2}, c_0 = \bot\)      & & X & & & & X & & X & & & & \\ \hline
 32 & \(a_1 = v_{a, 1}, b_2 = v_{b, 2}, c_1 = v_{c, 1}\)  & & X & & & & X & & & X & & & \\ \hline
 33 & \(a_1 = v_{a, 1}, b_2 = v_{b, 2}, c_2 = v_{c, 2}\)  & & X & & & & X & & & & X & & \\ \hline
 34 & \(a_1 = v_{a, 1}, b_2 = v_{b, 2}, c_3 = v_{c, 3}\)  & & X & & & & X & & & & & X & \\ \hline
 35 & \(a_1 = v_{a, 1}, b_2 = v_{b, 2}, c_4 = v_{c, 4}\)  & & X & & & & X & & & & & & X \\ \hline
%-----------------
 36 & \(a_1 = v_{a, 1}, b_3 = v_{b, 3}, c_0 = \bot\)      & & X & & & & & X & X & & & & \\ \hline
 37 & \(a_1 = v_{a, 1}, b_3 = v_{b, 3}, c_1 = v_{c, 1}\)  & & X & & & & & X & & X & & & \\ \hline
 38 & \(a_1 = v_{a, 1}, b_3 = v_{b, 3}, c_2 = v_{c, 2}\)  & & X & & & & & X & & & X & & \\ \hline
 39 & \(a_1 = v_{a, 1}, b_3 = v_{b, 3}, c_3 = v_{c, 3}\)  & & X & & & & & X & & & & X & \\ \hline
 40 & \(a_1 = v_{a, 1}, b_3 = v_{b, 3}, c_4 = v_{c, 4}\)  & & X & & & & & X & & & & & X \\ \hline
%------------------

\end{tabular}
\caption{Keys presented to aggregator}
\label{example_keys_to_aggregator}
\end{table}
}

We now explain the \(-1\) in the earlier formula. This is because 
the first key is a ``match-all'' and hence will be disregarded. 
Let's take a look at the second key 
 \((a_0 = \bot, b_0 = \bot, c_1 = v_{c, 1})\). Let ``price'' be a measure
 attribute. Then, passing this to the aggregator is equivalent to wanting the
 answer to the SQL statement
 \begin{verbatim}
SELECT SUM(price) FROM in_tbl WHERE C = V_C_1
 \end{verbatim}
One more example to drive home the point. Consider Row 40 with key 
\((a_1 = v_{a, 1}, b_3 = v_{b, 3}, c_4 = v_{c, 4})\). The equivalent SQL would be 

 \begin{verbatim}
SELECT SUM(price) FROM in_tbl WHERE A = V_A_1 AND B = V_B_3 AND C = V_C_4
 \end{verbatim}

\section{Implementation Details}

We now discuss how to create a simple and efficient implementation in C.

\begin{table}[hbtp]
\centering
\begin{tabular}{|l|l|l|} \hline \hline
{\bf Logical } & {\bf C} \\ \hline 
Number of raw attributes & {\tt int nR;} \\ \hline
Number of derived attributes per raw attribute & \verb+ int *nDR; /* [n] */+ \\ \hline
Number of inputs & {\tt int nX;} \\ \hline
Table~\ref{example_input_to_agg} & \verb+uint8_t *X; /*[nX] */+ \\ \hline
Key for aggregator & {\tt int key; } \\ \hline
\end{tabular}
\caption{Data Structures for C}
\label{data_structures_for_C}
\end{table}

Things to note
\bi
\item \(nX = \sum_i \mathrm{nDR}[i]\)
\item We assume that \(nX \leq 16\). This allows us to encode a derived
attribute in 4 bits. 
\item We assume that the maximum number of values that any derived attribute can
take on is 16. This allows ut to encode the value in 4 bits.
\item Hence, every row of Table~\ref{example_input_to_agg} can be encoded using
an 8 bit unsigned integer.
\item It is relatively simple to relax these constraints. However, one must
recognize that allowing unconstrained relaxation is meaningless since it really
doesn't make sense to do a group by when the number of values of the attribute
is very high e.g., one would never say ``group by time since epoch in seconds''
\item We assume that \(nR \leq 4\). Once again, this is easy to relax but the
assumption serves us well for now.
\ei


\bibliographystyle{alpha}
\bibliography{../../Q_PAPER/ref} 
