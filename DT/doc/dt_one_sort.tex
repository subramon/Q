\documentclass[12pt,letterpaper]{article}
\usepackage{times}
\usepackage{helvet}
\usepackage{courier}
\usepackage{hyperref}
\usepackage{fancyheadings}
\pagestyle{fancy}
\usepackage{pmc}

%%%%%%%%%%%%%%%%%%%%%%%%%%%%%%%%%%%%%%%%%%%%%%%%%%%%%%%%%%%%%%%%%%%%%%%%%%%%%%%%%%
% Special for e-unibus doc commands

\newcommand{\ForLater}{
\begin{center}
{\bf NOT FOR CURRENT VERSION}
\end{center}
}
\newcommand{\TBC}{\framebox{\textbf{TO BE COMPLETED}}}
\newcommand{\DISCUSS}{\Ovalbox {\bf \textcolor{red}{FOR DISCUSSION}}}
\newcommand{\Input}{\framebox{\textsf{in}}}
\newcommand{\Output}{\framebox{\textsf{out}}}
\newcommand{\debug}[1]{\textbf{debug start} #1 \textbf{debug finish}}
\newcommand{\inx}[1]{\emph{#1}}
\newtheorem{notation}{Notation}
\newtheorem{definition}{Definition}
\newtheorem{problem_statement}{Problem Statement}
\newtheorem{invariant}{Invariant}
\newtheorem{assumption}{Assumption}
\newtheorem{resource_string}{Resource String}
\newtheorem{testcase}{Test Case}
\newtheorem{note}{Note}
\newtheorem{specification}{Specification}
\newtheorem{caution}{Caution}
\newtheorem{prereq}{Pre-requisite}
\newtheorem{action}{Action}
\newtheorem{query}{Query}
\newcommand{\be}{\begin{enumerate}}
\newcommand{\ee}{\end{enumerate}}
\newcommand{\bi}{\begin{itemize}}
\newcommand{\ei}{\end{itemize}}
\newcommand{\bv}{\begin{verbatim}}
\newcommand{\ev}{\end{verbatim}}
\newcommand{\bd}{\begin{description}}
\newcommand{\ed}{\end{description}}
\newcommand{\bpre}{\begin{prereq}}
\newcommand{\epre}{\end{prereq}}
\newcommand{\bact}{\begin{action}}
\newcommand{\eact}{\end{action}}
\newcommand{\bs}{\begin{specification}}
\newcommand{\es}{\end{specification}}
\newcommand{\btc}{\begin{testcase}}
\newcommand{\etc}{\end{testcase}}
\newcommand{\bc}{\begin{caution}}
\newcommand{\ec}{\end{caution}}
\newcommand{\la}{\leftarrow}
\newcommand{\IpArgs}{\subsection{Input Arguments}}
\newcommand{\PreReqs}{\subsection{Pre-requisities}}
\newcommand{\Actions}{\subsection{Actions}}
\newcommand{\Coverage}{{\bf To test coverage.}}

%%%%%%%%%%%%%%%%%%%%%%%%%%%%%%%%%%%%%%%%%%%%%%%%%%%%%%%%%%%%%%%%%%%%%%%%%%%


\newtheorem{theorem}{Theorem}[section]
\newtheorem{lemma}[theorem]{Lemma}
\newtheorem{proposition}[theorem]{Proposition}
\newtheorem{corollary}[theorem]{Corollary}

\newenvironment{proof}[1][Proof]{\begin{trivlist}
\item[\hskip \labelsep {\bfseries #1}]}{\end{trivlist}}
\newenvironment{intuition}[1][Intuition]{\begin{trivlist}
\item[\hskip \labelsep {\bfseries #1}]}{\end{trivlist}}
%% \newenvironment{definition}[1][Definition]{\begin{trivlist}
%% \item[\hskip \labelsep {\bfseries #1}]}{\end{trivlist}}
\newenvironment{example}[1][Example]{\begin{trivlist}
\item[\hskip \labelsep {\bfseries #1}]}{\end{trivlist}}
\newenvironment{remark}[1][Remark]{\begin{trivlist}
\item[\hskip \labelsep {\bfseries #1}]}{\end{trivlist}}

\newcommand{\qed}{\nobreak \ifvmode \relax \else
      \ifdim\lastskip<1.5em \hskip-\lastskip
      \hskip1.5em plus0em minus0.5em \fi \nobreak
      \vrule height0.75em width0.5em depth0.25em\fi}

%%%%%%%%%%%%%%%%%%%%%%%%%%%%%%%%%%%%%%%%%%%%%%%%%%%%%%%%%%%%%%%%%
% \newcommand{\Alogon}{\mbox{\fontfamily{ptm}\selectfont {\large \selectfont A} \hspace{-1.2ex} {\large \selectfont L} \hspace{-2.3ex} \raisebox{0.45ex}{ {\footnotesize \selectfont O} } \hspace{-1.80ex} {\large \selectfont G} \hspace{-1.80ex} \raisebox{-0.33ex}{ {\large \selectfont O} } \hspace{-1.8ex} {\large \selectfont N}}}


\usepackage{graphicx}
\setlength\textwidth{6.5in}
\setlength\textheight{9.0in}
\begin{document}
\title{Computing Decision Trees with a Single Sort}
\author{Tara Mirmira and Ramesh Subramonian}
\maketitle
\thispagestyle{fancy}
\lhead{}
\chead{}
\rhead{}
\lfoot{}
\cfoot{Decision Trees}
\rfoot{{\small \thepage}}

\begin{abstract}
A typical computational strategy for building decision trees (as evidenced in
  scikit-learn) is as follows. Each feature is sorted and then traversed in
  order to determine the best split point. The best split point over
  all features is selected and used to partition the data into two. This
  process is repeated recursively until some stopping criterion is reached e.g.,
  the number of instances is
  too small. The contribution of this paper is to provide a novel indexing
  strategy that requires only a single sort at the beginning. After that, no
  further sorting is required. A linear scan of the data, with the values
  reordered and the indexes updated as we go, is sufficient to maintain the sorted order.
\end{abstract}

\section{Introduction}

\TBC

\subsection{Motivation}
Notwithstanding the growing popularity of neural networks, decision trees and
random forests continue to be a widely used machine learning technique.
Efficient training of decision trees becomes important when 
\be
\item the data set sizes become large
\item there is a large hyper-parameter space over which to tune
\item a random forest may employ a large number of decision  trees
\ee

This calls for an efficient algorithm and implementation for the kernel of the
computation. This can be summarized as 
\be
\item scanning all the distinct values of all the features to see how they
  partition the data in terms of the goal attribute
\item computing the metric of interest (gini impurity, entropy, \ldots) for 
 each of the
  candidate split points
  \ee

\section{Algorithm}
\subsection{Assumptions}
Our current implementation makes the following assumptions. These are purely for
the sake of convenience --- these are not conceptual limitations inherent in
the approach.

\be
\item The goal attribute can be encoded as 0 or 1
\item The values of the attributes used to build the decision tree are ordered and can be
  represented as floating point numbers
\item \(n \leq 2^{32}\), where \(n\) is the number of instances
\item The number of unique values for each instance is \(< 2^{31}\)
\item There are no missing values. 
\ee

\subsection{Data Structures}

\be
\item Let \(n\) be the number of instances
\item Let \(m\) be the number of features. 
\item Let \(X[m][n]\) be the input data. \(X[j]\), also denoted as \(X_j\),  is a column vector containg
  the \(n\) values of feature \(j\)
\item Let \(g[n]\) be the values of the goal attribute
\item Let \(Y[m][n]\) be the transformed data where input data has been
  ``position-encoded'' 
  by its position in the sort order (ascending). A sample mapping from 
  X-values to Y-values is shown below.
\begin{displaymath}
[11, 32, 47, 11, 17, 28, 32, 55] \Rightarrow [1, 4, 5, 1, 2, 4, 6]
\end{displaymath}

Further, for efficiency, each element of Y is encoded as a 64-bit integer where 
\bi
\item bits \([0..30]\) represent the Y-value itself. 
  We refer to this as \(Y_j[i].y\)
\item bit 31 represent the goal value
  We refer to this as \(Y_j[[i].g\)
\item bits \([32..63]\) represent the {\bf from} value, explained later
  We refer to this as \(Y_j[i].f\)
  \ei
\(Y_j\) is sorted so that \(Y_j[i].y \leq Y_j[i+1].y\)
In order to record the original position of this value, we use the {\bf from}
field. The inter-relationship is specified as follows:
\bi
\item \(x = X_j[k]\)
\item \(k = Y_j[i].f\)
\item \(y = Y_j[i].y\)
\item Then, \(y\) is the position-encoded value of \(x\)
  \ei

\item Let \(T[m][n]\) be a data structure used to record the ``to'' indexing.
  Intuitively, it tells us {\bf to} which position a datum in the original set
  has been permuted. Its interlinking with the {\bf from} field is best explained
  in Invariant~\ref{from_to}
  \ee

\begin{invariant}
  \label{from_to}
  Let \(p = Y_j[i].f\). Then \(T_j[p] = i\)

\end{invariant}

The algorithm is motivated with a simple example where
\be
\item \(n= 16\)
\item \(m= 2\)
\item The \(x\)-values and their corresponding position-encoded \(y\)-values are the same
\item Column {\bf P} represents position
\item Column {\bf L} is a label (not used by the algorithm)
\item Column \(F_1\) is the values of feature 1 
\item Column \(F_2\) is the values of feature 2
\item Column {\bf G} is the values of the goal
\item Column \(Y_1\) is the sorted, encoded values of \(F_1\). Values in the column
  are a tuple \((f,g,y)\), where 
  \be
\item \(f\) --- the position of this value in the original data set, F1.
\item \(g\) --- the value of the goal feature for this instance
\item \(y\) --- the position-encoded value for this feature
  \ee
\item Column \(Y_2\) is the sorted, encoded values of \(F_2\)
\item Column \(T_1\) is the {\bf to} data structure for \(F_1\)
\item Column \(T_2\) is the {\bf to} data structure for \(F_2\)
  \ee

  \begin{table}
    \begin{tabular}{|l|l|l|l|l|l|l|l|l|l|l|l|l|l|l|l|} \hline \hline
      {\bf P} & {\bf L} & \(F_1\) & \(F_2\) & {\bf G} & \(Y_1\) &  %
      \(Y_2\) & \(T_1\) & \(T_2\) & \(Y'_2\) & \(T'_2\) \\ \hline \hline
      0 & A & 8 & 7 & 1 & (3 , 0 , 1) & (13 , 1 , 1) & 7 & 6 & (13 , 1 , 1) & 2 \\ \hline
1 & B & 3 & 2 & 0 & (15 , 1 , 2) & (1 , 0 , 2) & 2 & 1 & (1 , 0 , 2) & 1 \\ \hline
2 & C & 12 & 11 & 0 & (1 , 0 , 3) & (14 , 0 , 3) & 11 & 10 & (0 , 1 , 7) & 13 \\ \hline
3 & D & 1 & 14 & 0 & (13 , 0 , 4) & (7 , 0 , 4) & 0 & 13 & (8 , 1 , 8) & 6 \\ \hline
4 & E & 10 & 9 & 1 & (5 , 1 , 5) & (11 , 1 , 5) & 9 & 8 & (15 , 1 , 10) & 12 \\ \hline
5 & F & 5 & 16 & 1 & (8 , 1 , 6) & (12 , 1 , 6) & 4 & 15 & (9 , 0 , 12) & 7 \\ \hline
6 & G & 13 & 15 & 1 & (9 , 1 , 7) & (0 , 1 , 7) & 12 & 14 & (3 , 0 , 14) & 15 \\ \hline
7 & H & 11 & 4 & 0 & (0 , 0 , 8) & (8 , 1 , 8) & 10 & 3 & (5 , 1 , 16) & 9 \\ \hline
8 & I & 6 & 8 & 1 & (10 , 1 , 9) & (4 , 1 , 9) & 5 & 7 & (14 , 0 , 3) & 3 \\ \hline
9 & J & 7 & 12 & 0 & (4 , 0 , 10) & (15 , 1 , 10) & 6 & 11 & (7 , 0 , 4) & 5 \\ \hline
10 & K & 9 & 13 & 1 & (7 , 1 , 11) & (2 , 0 , 11) & 8 & 12 & (11 , 1 , 5) & 14 \\ \hline
11 & L & 14 & 5 & 1 & (2 , 1 , 12) & (9 , 0 , 12) & 13 & 4 & (12 , 1 , 6) & 10 \\ \hline
12 & M & 15 & 6 & 1 & (6 , 1 , 13) & (10 , 1 , 13) & 14 & 5 & (4 , 1 , 9) & 11 \\ \hline
13 & N & 4 & 1 & 1 & (11 , 1 , 14) & (3 , 0 , 14) & 3 & 0 & (2 , 0 , 11) & 0 \\ \hline
14 & O & 16 & 3 & 0 & (12 , 0 , 15) & (6 , 1 , 15) & 15 & 2 & (10 , 1 , 13) & 8 \\ \hline
15 & P & 2 & 10 & 1 & (14 , 1 , 16) & (5 , 1 , 16) & 1 & 9 & (6 , 1 , 15) & 4 \\ \hline

      \hline
    \end{tabular}
  \end{table}

\subsection{Example}

Let us assume that the best split is using feature \(F_1\) such that the 
first \(i_L = 8\) 8
elements are assigned to the left sub-tree and the remaining to the rest.
In other words, \(F_1 \leq 8\) selects the data for the left child, with the
balance to the right.
\bi
\item The instances for the left sub-treee have labels 
  \(\{D, P, B, N, F, I, J, A\}\)
\item The instances for the right sub-treee have labels 
  \(\{K, E, H, C, G, L, M, O\}\)
\ei
When we move to processing the left and right sub-trees, no
additional processing needs to be done for feature \(F_1\) because it already in the the correct sorted order. 
However,  \(F_2\) needs to be re-arranged so that it is in sorted order. This is
done as follows
\be
\item traverse the \(Y_2\) column in order. Use \(p = Y_2[i].f\) to figure out where
this item came {\em from}. Use \(q = T_1[p]\) to decide where this item went {\bf
to}. If \(q \leq i_L = 8\), then we know that it goes to the left sub-tree; else
it goes to the right sub -tree. Since the \(Y_2\) column was sorted, and we
build the left and right trees by scanning \(Y_2\) in order, the left
and right trees continue to be in sorted order. 
This allows us to build the \(Y'_2\) column. 
\item 
There is one last detail to consider. Which is that the \(T'_2\) field needs to
be updated to reflect the reordering of the values of \(F_2\) in \(Y'_2\).

This is done as follows. Assume \(Y_2[i]\) is to be moved to positon \(k\). Let
\(f = Y_2[i].f\). Then, \(T'_2[f] \leftarrow k\)
\ee

Note that we need to keep \(Y_2, T_2\) until we have built \(Y'_2, T'_2\) at
which point we can replace them with \(Y'_2, T'_2\)

As a quick sanity check, notice that the values of \(F_2\) in left and right
sub-trees are sorted as:
\begin{description}
  \item [left] 
\([1, 2, 7, 8, 10, 12, 14, 16]\), see rows 0 to 7 of column \(Y'_2\).
\item [right] 
\([3, 4, 5, 6, 9, 11, 13, 15]\) , see rows 8 to 15 of column \(Y'_2\).
\end{description}

\subsection{The Algorithm}

The key insight that drives the algorithm is that preserving the sorted order of
the features {\em not} chosen for the current split does {\em not} require a
re-sort. Instead, as long as we can identify whether an element should be
assigned to the left or right sub-tree, then we can scan the elements in order
and add them to the left/right sub-trees. This preserves the sorted order and
allows us to maintain the doubly-indexed data structures and sets us up for a
recursive solution.


\section{Conclusion}

\TBC

\bibliographystyle{alpha}
\bibliography{../../DOC/Q_PAPER/ref}
\end{document}
