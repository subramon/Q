\documentclass[12pt,letterpaper]{article}
\usepackage{times}
\usepackage{helvet}
\usepackage{courier}
\usepackage{hyperref}
\usepackage{fancyheadings}
\pagestyle{fancy}
\usepackage{pmc}

%%%%%%%%%%%%%%%%%%%%%%%%%%%%%%%%%%%%%%%%%%%%%%%%%%%%%%%%%%%%%%%%%%%%%%%%%%%%%%%%%%
% Special for e-unibus doc commands

\newcommand{\ForLater}{
\begin{center}
{\bf NOT FOR CURRENT VERSION}
\end{center}
}
\newcommand{\TBC}{\framebox{\textbf{TO BE COMPLETED}}}
\newcommand{\DISCUSS}{\Ovalbox {\bf \textcolor{red}{FOR DISCUSSION}}}
\newcommand{\Input}{\framebox{\textsf{in}}}
\newcommand{\Output}{\framebox{\textsf{out}}}
\newcommand{\debug}[1]{\textbf{debug start} #1 \textbf{debug finish}}
\newcommand{\inx}[1]{\emph{#1}}
\newtheorem{notation}{Notation}
\newtheorem{definition}{Definition}
\newtheorem{problem_statement}{Problem Statement}
\newtheorem{invariant}{Invariant}
\newtheorem{assumption}{Assumption}
\newtheorem{resource_string}{Resource String}
\newtheorem{testcase}{Test Case}
\newtheorem{note}{Note}
\newtheorem{specification}{Specification}
\newtheorem{caution}{Caution}
\newtheorem{prereq}{Pre-requisite}
\newtheorem{action}{Action}
\newtheorem{query}{Query}
\newcommand{\be}{\begin{enumerate}}
\newcommand{\ee}{\end{enumerate}}
\newcommand{\bi}{\begin{itemize}}
\newcommand{\ei}{\end{itemize}}
\newcommand{\bv}{\begin{verbatim}}
\newcommand{\ev}{\end{verbatim}}
\newcommand{\bd}{\begin{description}}
\newcommand{\ed}{\end{description}}
\newcommand{\bpre}{\begin{prereq}}
\newcommand{\epre}{\end{prereq}}
\newcommand{\bact}{\begin{action}}
\newcommand{\eact}{\end{action}}
\newcommand{\bs}{\begin{specification}}
\newcommand{\es}{\end{specification}}
\newcommand{\btc}{\begin{testcase}}
\newcommand{\etc}{\end{testcase}}
\newcommand{\bc}{\begin{caution}}
\newcommand{\ec}{\end{caution}}
\newcommand{\la}{\leftarrow}
\newcommand{\IpArgs}{\subsection{Input Arguments}}
\newcommand{\PreReqs}{\subsection{Pre-requisities}}
\newcommand{\Actions}{\subsection{Actions}}
\newcommand{\Coverage}{{\bf To test coverage.}}

%%%%%%%%%%%%%%%%%%%%%%%%%%%%%%%%%%%%%%%%%%%%%%%%%%%%%%%%%%%%%%%%%%%%%%%%%%%


\newtheorem{theorem}{Theorem}[section]
\newtheorem{lemma}[theorem]{Lemma}
\newtheorem{proposition}[theorem]{Proposition}
\newtheorem{corollary}[theorem]{Corollary}

\newenvironment{proof}[1][Proof]{\begin{trivlist}
\item[\hskip \labelsep {\bfseries #1}]}{\end{trivlist}}
\newenvironment{intuition}[1][Intuition]{\begin{trivlist}
\item[\hskip \labelsep {\bfseries #1}]}{\end{trivlist}}
%% \newenvironment{definition}[1][Definition]{\begin{trivlist}
%% \item[\hskip \labelsep {\bfseries #1}]}{\end{trivlist}}
\newenvironment{example}[1][Example]{\begin{trivlist}
\item[\hskip \labelsep {\bfseries #1}]}{\end{trivlist}}
\newenvironment{remark}[1][Remark]{\begin{trivlist}
\item[\hskip \labelsep {\bfseries #1}]}{\end{trivlist}}

\newcommand{\qed}{\nobreak \ifvmode \relax \else
      \ifdim\lastskip<1.5em \hskip-\lastskip
      \hskip1.5em plus0em minus0.5em \fi \nobreak
      \vrule height0.75em width0.5em depth0.25em\fi}

%%%%%%%%%%%%%%%%%%%%%%%%%%%%%%%%%%%%%%%%%%%%%%%%%%%%%%%%%%%%%%%%%
% \newcommand{\Alogon}{\mbox{\fontfamily{ptm}\selectfont {\large \selectfont A} \hspace{-1.2ex} {\large \selectfont L} \hspace{-2.3ex} \raisebox{0.45ex}{ {\footnotesize \selectfont O} } \hspace{-1.80ex} {\large \selectfont G} \hspace{-1.80ex} \raisebox{-0.33ex}{ {\large \selectfont O} } \hspace{-1.8ex} {\large \selectfont N}}}


\usepackage{graphicx}
\setlength\textwidth{6.5in}
\setlength\textheight{9.0in}
% \newcommand{\bd}{\begin{description}}
% \newcommand{\ed}{\end{description}}
\begin{document}
\title{When and where to rebuild a decision tree \\
with constant space/time overhead}
\author{Tara Mirmira and Ramesh Subramonian}
\maketitle
\thispagestyle{fancy}
\lhead{}
\chead{}
\rhead{}
\lfoot{}
\cfoot{Decision Trees}
\rfoot{{\small \thepage}}

\begin{abstract}
I Incremental/Decremental on Decision Trees
\end{abstract}

\section{Introduction}

\TBC

\subsection{Motivation}
\TBC

\section{Algorithm}

\subsection{Intuition}

The key intuition behind this algorithm is that the cost of reordering is
significantly less
than the cost of the computation of the metric (e.g., gini) used to determine
the best split. This allows us to store just one additional integer for 
each data value and yet be able to efficiently determine whether and where retraining is
required. For a data set with \(m\) features and \(n\) instances, we need \(m
\times n\) additional integers. In contrast, \TBC \cite{Brophy2021}
% The size of the integers depends on \(n\). 
% So, a 32-bit integer is sufficient when \(n \leq 2^{32}\).
\subsection{Assumptions}

Our current implementation makes the following assumptions. These are purely for
the sake of convenience --- these are not conceptual limitations inherent in
the approach.

\be
\item The goal attribute can be encoded as 0 or 1. We
  refer to the 0 value as Tails and the 1 value as Heads.
% \item The maximum depth of the decision tree is 64
\item There are no missing values. 
\ee

\subsection{Definitions}
\be
\item Let \(n\) be the number of instances or data points
\item Let \(n^H\) be the number of instances with goal value = 1 (Heads)
\item Let \(n^T\) be the number of instances with goal value = 0 (Tails)
\item Clearly, \(n = n^H+n^T\)
\item Let \(m\) be the number of features or attributes
\ee

\begin{definition}
\label{decision_path}
We define \(D(x, T) = \{l_0, l_1, \ldots\}\) the {\bf decision path} of a data point \(x\) with
  respect to a tree \(T\) as the sequence of decisions made, starting from the root and ending up at a leaf.
  \TBC
\end{definition}


\begin{definition}
\label{leaf_node}
We define \(L(x, T)\), the {\bf leaf node} of a data point, \(x\) with 
respect to a tree \(T\) as the leaf node at which it lands up \TBC 
\end{definition}

Cleatly, for all points \(x\), the first node of the decision path 
is the root and the last node of the path is the leaf node, \(L(x, T)\)


We will use the term virtual columns to denote columns that are not
materialized. However, assuming that they do exist simplifies the description of
the algorithm.


\subsection{Data Structures}
There are three principal data structures we use ---  \(X, Y, T\).
The example in Section~\ref{example_data_structures} will hopefully make the
formal description less daunting.
\subsubsection{X}
This represents the input data, which is 
stored in columnar fashion. The columns (of length \(n\)) are:
  \be
\item \(z\), deletion flag. \(X[i].z = \mathrm{false} \Rightarrow X[i]\) has been
  deleted. 
\item \(i\), representing the index of the data point--- values are \(0, 1,
  \ldots n-1\) . Virtual
\item \(X_1, X_2, \ldots X_m\) representing the \(m\) features.
  So, \(X_j[i]\) is the value of the \(j^{th}\) feature of the \(i^{th}\) data
  point.
\item \(g\), goal attribute
\item \(l\), the leaf node (Definition~\ref{leaf_node}). 
  {\bf QUESTION:} {\em Should this be a virtual column?}
\ee

\subsubsection{Y}
\(Y\) consists of \(m\) data structures, \(\{Y_j\}\). \(Y_j\)
represents the data for feature \(j\) stored as a set of columns 
(of length \(n\)). These are 
\be
\item \(z\), deletion flag, boolean, false means deleted. Virtual
\item \(v\), feature value. Virtual
\item \(g\), goal value. Virtual
\item \(l\), leaf node. Virtual
\item \(h\), cumulative number of positive instances, run time computation
\item \(t\), cumulative number of negative instances, run time computation
\item \(i\), index belonging to corresponding data point
\ee
The explanation of the above columns is provided below:
\be
\item \(Y_j[i].v \leq Y_j[i+1].v\)
\item \(Y_j.i\) is a permutation of \(0, \ldots, n-1\)
\item Virtual columns are created as follows, where \(Y_j[k].i = k'\)
  \be
\item \(Y_j[k].v = X_j[k']\) 
\item \(Y_j[k].g = X.g[k']\) 
\item \(Y_j[k].z = X.z[k']\) 
\item \(Y_j[k].l = X.l[k']\) 
\item \(Y_j.h, Y_j.t\) are constructed to give us is the number of 
Heads and Tails to the left of each data point for feature \(j\), as shown in
Figure~\ref{code_nH_nT}
\ee
\ee


\begin{figure}
\centering
\fbox{
  \begin{minipage}{8 cm}
  \centering
  \begin{tabbing} \hspace*{0.25in} \= \hspace*{0.25in} \= %
    \hspace*{0.25in} \= \kill
Let \(k' = Y_j[k].i\) \\
    Let \(h[k] \leftarrow 1\) if (\(X.g[k'] = 1\) {\bf and} \(X.z[k'] = 1\)) and 0 otherwise \\
    Let \(h[k] \leftarrow 1\) if (\(X.g[k'] = 1\) {\bf and} \(X.z[k'] = 1\)) and 0 otherwise \\
\(Y_j.h[0] \leftarrow h[0]\) \\
\(k > 0 \Rightarrow Y_j.h[k] \leftarrow Y_j.h[k-1] + h[k]\) \\
    Let \(t[k] \leftarrow 1\) if (\(X.g[k'] = 0\) {\bf and} \(X.z[k'] = 1\)) and 0 otherwise \\
\(Y_j.t[0] \leftarrow t[0]\) \\
\(k > 0 \Rightarrow Y_j.t[k] \leftarrow Y_j.t[k-1] + t[k]\) 
  \end{tabbing}
    \caption{Pseudo code for computing \(Y_j.h, Y_j.t\)}
  \label{code_nH_nT}
  \end{minipage}
}
\end{figure}

\subsubsection{T}
This is the decision tree represented as an array, where each element has the
following fields 
  \be
\item \(z\), deletion flag, boolean, false means deleted.
\item \(p\), parent
\item \(l\), left child 
\item \(r\), left child 
\item \(n\), number of data instances whose decision path includes this node.
  Before any data items  are deleted, \(Z[0].n = n\) indicating that {\bf all}
  data points pass through the root.
\item \(d\), depth. \(Z[0].d = 0\) indicating that the root is at depth 0 
\item \(m\), best metric 
\item \(f\), feature with best metric 
\item \(v\), best split value of feature with best metric 
% \item \(P\), the path taken by this data point from the root to the
%   leaf node, stored as a bit mask. For example, say \(d = 3\) and the
%   bit-pattern
%   of \(P = 010\). Then, reading the bit pattern from least significant bit to
%   most significant bit, this node is the left child of the right child of the left child of the root.
  \ee

\subsubsection{An example}
\label{example_data_structures}

An example of the inter-relationship between \(X\) and \(Y\) is provided in
Table~\ref{tbl_example_data_structures}.

\begin{table}
  \centering
  \begin{tabular}{|l|l|l|l|l||l|l|l|l||l|l|l|l|} \hline \hline
    \(i\) & \(X_1\) & \(X_2\) & \(g\) & \(z\) & %
    \(Y_1.i\) & \(Y_1.v\) & \(Y_1.h\) & \(Y_1.t\) & 
    \(Y_2.i\) & \(Y_2.v\) & \(Y_2.h\) & \(Y_2.t\) \\ \hline \hline
    0 & 4  & 1  & 1 & 1 & 4 & 1  & 0 & 0 & 0 & 1  & 1 & 0 \\ \hline
    1 & 9  & 9  & 1 & 1 & 9 & 2  & 0 & 0 & 7 & 2  & 1 & 1 \\ \hline
    2 & 5  & 4  & 0 & 1 & 5 & 3  & 0 & 1 & 5 & 3  & 1 & 2 \\ \hline
    3 & 7  & 6  & 1 & 1 & 0 & 4  & 1 & 1 & 2 & 4  & 1 & 3 \\ \hline
    4 & 1  & 5  & 1 & 0 & 2 & 5  & 1 & 2 & 4 & 5  & 1 & 3 \\ \hline
    5 & 3  & 3  & 0 & 1 & 8 & 6  & 1 & 3 & 3 & 6  & 2 & 3 \\ \hline
    6 & 10 & 7  & 0 & 1 & 3 & 7  & 2 & 3 & 6 & 7  & 2 & 4 \\ \hline
    7 & 8  & 2  & 0 & 1 & 7 & 8  & 2 & 4 & 9 & 8  & 2 & 4 \\ \hline
    8 & 6  & 10 & 0 & 1 & 1 & 9  & 3 & 4 & 1 & 9  & 3 & 4 \\ \hline
    9 & 2  & 8  & 1 & 0 & 6 & 10 & 3 & 5 & 8 & 10 & 3 & 5 \\ \hline
    \hline
  \end{tabular}
  \caption{Example of Data Structures}
\label{tbl_example_data_structures}
\end{table}

\section{Deletion}
We first provide the intuition behind the algorithm.  Let's describe what 
happens upon deletion of a data point with index \(i\). 
We start by considering the impact of this deletion on node \(l' = 0\), the
root.

We use \(Y\) to evaluate the best metric for each feature, 
{\em after} this deletion has been factored in. This is described in 
Section~\ref{Metric_Computation}.

If the new best metric improves upon \(T[l'].m\), 
the sub-tree rooted at \(l'\) needs to be discarded and recomputed and grafted
back on to the original tree. This is described in Section~\ref{Grafting}.

If none of the features realize a better metric, then we determine whether this
data point would have next visited the left child or the right child of \(l'\). 

need to recursively
perform the same test at node \(l_{i+1}\). 
In order to make the recursion possible, we need to
create \(Y'\) from \(Y\). This is done by 
\(Y' \leftarrow Create(Y, l_{i+1}\), described in Section~\ref{ReCreate_Y}.

The recursion stops when a sub-tree needs to be recomputed or we arrive at a
leaf. 

\section{Metric Computation}
\label{Metric_Computation}

Let \(h = Y_j.h, t = Y_j.t\) for some feature \(j\). 
Figure~\ref{code_metric_comp} shows how
the gini metric is computed for each possible split point. 
\begin{figure}
\centering
\fbox{
  \begin{minipage}{8 cm}
  \centering
  \begin{tabbing} \hspace*{0.25in} \= \hspace*{0.25in} \= %
    \hspace*{0.25in} \= \kill
    {\bf function} \(\mathrm{BestGini}(h, t)\) \+ \\
      \(n^H_L = h[0], n^T_L = t[0]\) \\
      \(m = \mathrm{Gini}(n^H_L, n^T_L, n^H, n^T)\) \\ 
      {\bf for} \(i = 1\) {\bf to} \(n-1\) {\bf do} \+ \\
        \(n^H_L = n^H_L + h[i], n^T_L = n^T_L + h[i]\) \\ 
        \(m = \mathrm{min}(m, \mathrm{Gini}(n^H_L, n^T_L, n^H, n^T))\) \- \\
     {\bf endfor} \\ 
     {\bf return} \(m\) \- \\
   {\bf end} \\
  \end{tabbing}
  \caption{Pseudo code for metric computation}
  \label{code_metric_comp}
  \end{minipage}
}
\end{figure}

Gini computation is in Figure~\ref{gini_comp}
\begin{figure}
\centering
\fbox{
  \begin{minipage}{8 cm}
  \centering
  \begin{tabbing} \hspace*{0.25in} \= \hspace*{0.25in} \= %
    \hspace*{0.25in} \= \kill
    {\bf function} \(\mathrm{Gini}( n^H_L, n^T_L, n^H, n^T))\) \+ \\
      \TBC \\
      {\bf return} \(x\) \- \\
    {\bf end} \\
  \end{tabbing}
  \caption{Pseudo code for gini computation}
  \label{gini_comp}
  \end{minipage}
}
\end{figure}

\section{Re-Create Y}
\label{ReCreate_Y}

See Figure~\ref{pseudo_code_recreate_Y}.
\begin{figure}
\centering
\fbox{
  \begin{minipage}{12 cm}
   \centering
   \begin{tabbing} \hspace*{0.25in} \= \hspace*{0.25in} \= %
     \hspace*{0.25in} \= \kill
     {\bf function} \(\mathrm{CreateY}(l, Y)\) \+ \\ 
       Let \(n_Y\) be size of \(Y\) \\ 
       Allocate space for \(Y'\). Space needed = \(n_Y'= T[l].n\) \\
       {\bf for} \(i \leftarrow 0\) {\bf to} \(n_Y-1\) {\bf do} \+ \\
           \TBC \- \\ 
       {\bf endfor} \\
       {\bf return} \(Y'\) \- \\
     {\bf end} \\
   \end{tabbing}
     \caption{Pseudo code for creating Y on recursive step}
   \label{pseudo_code_recreate_Y}
  \end{minipage}
}
\end{figure}


\section{Grafting}
\label{Grafting}

the sub-tree rooted at \(l\) needs to be discarded (\(z \leftarrow
\mathrm{false}\) for all such nodes) `and replaced by a
new sub-tree built using all points with index \(i\) such that \(X[i].l\) is a descendant of
\(l'\)

\section{Conclusion}

\TBC

\bibliographystyle{alpha}
\bibliography{../../DOC/Q_PAPER/ref}
\end{document}
