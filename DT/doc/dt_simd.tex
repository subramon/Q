\documentclass[12pt,letterpaper]{article}
\usepackage{times}
\usepackage{helvet}
\usepackage{courier}
\usepackage{hyperref}
\usepackage{fancyheadings}
\pagestyle{fancy}
% TODO \usepackage{pmc}

%%%%%%%%%%%%%%%%%%%%%%%%%%%%%%%%%%%%%%%%%%%%%%%%%%%%%%%%%%%%%%%%%%%%%%%%%%%%%%%%%%
% Special for e-unibus doc commands

\newcommand{\ForLater}{
\begin{center}
{\bf NOT FOR CURRENT VERSION}
\end{center}
}
\newcommand{\TBC}{\framebox{\textbf{TO BE COMPLETED}}}
\newcommand{\DISCUSS}{\Ovalbox {\bf \textcolor{red}{FOR DISCUSSION}}}
\newcommand{\Input}{\framebox{\textsf{in}}}
\newcommand{\Output}{\framebox{\textsf{out}}}
\newcommand{\debug}[1]{\textbf{debug start} #1 \textbf{debug finish}}
\newcommand{\inx}[1]{\emph{#1}}
\newtheorem{notation}{Notation}
\newtheorem{definition}{Definition}
\newtheorem{problem_statement}{Problem Statement}
\newtheorem{invariant}{Invariant}
\newtheorem{assumption}{Assumption}
\newtheorem{resource_string}{Resource String}
\newtheorem{testcase}{Test Case}
\newtheorem{note}{Note}
\newtheorem{specification}{Specification}
\newtheorem{caution}{Caution}
\newtheorem{prereq}{Pre-requisite}
\newtheorem{action}{Action}
\newtheorem{query}{Query}
\newcommand{\be}{\begin{enumerate}}
\newcommand{\ee}{\end{enumerate}}
\newcommand{\bi}{\begin{itemize}}
\newcommand{\ei}{\end{itemize}}
\newcommand{\bv}{\begin{verbatim}}
\newcommand{\ev}{\end{verbatim}}
\newcommand{\bd}{\begin{description}}
\newcommand{\ed}{\end{description}}
\newcommand{\bpre}{\begin{prereq}}
\newcommand{\epre}{\end{prereq}}
\newcommand{\bact}{\begin{action}}
\newcommand{\eact}{\end{action}}
\newcommand{\bs}{\begin{specification}}
\newcommand{\es}{\end{specification}}
\newcommand{\btc}{\begin{testcase}}
\newcommand{\etc}{\end{testcase}}
\newcommand{\bc}{\begin{caution}}
\newcommand{\ec}{\end{caution}}
\newcommand{\la}{\leftarrow}
\newcommand{\IpArgs}{\subsection{Input Arguments}}
\newcommand{\PreReqs}{\subsection{Pre-requisities}}
\newcommand{\Actions}{\subsection{Actions}}
\newcommand{\Coverage}{{\bf To test coverage.}}

%%%%%%%%%%%%%%%%%%%%%%%%%%%%%%%%%%%%%%%%%%%%%%%%%%%%%%%%%%%%%%%%%%%%%%%%%%%


\newtheorem{theorem}{Theorem}[section]
\newtheorem{lemma}[theorem]{Lemma}
\newtheorem{proposition}[theorem]{Proposition}
\newtheorem{corollary}[theorem]{Corollary}

\newenvironment{proof}[1][Proof]{\begin{trivlist}
\item[\hskip \labelsep {\bfseries #1}]}{\end{trivlist}}
\newenvironment{intuition}[1][Intuition]{\begin{trivlist}
\item[\hskip \labelsep {\bfseries #1}]}{\end{trivlist}}
%% \newenvironment{definition}[1][Definition]{\begin{trivlist}
%% \item[\hskip \labelsep {\bfseries #1}]}{\end{trivlist}}
\newenvironment{example}[1][Example]{\begin{trivlist}
\item[\hskip \labelsep {\bfseries #1}]}{\end{trivlist}}
\newenvironment{remark}[1][Remark]{\begin{trivlist}
\item[\hskip \labelsep {\bfseries #1}]}{\end{trivlist}}

\newcommand{\qed}{\nobreak \ifvmode \relax \else
      \ifdim\lastskip<1.5em \hskip-\lastskip
      \hskip1.5em plus0em minus0.5em \fi \nobreak
      \vrule height0.75em width0.5em depth0.25em\fi}

%%%%%%%%%%%%%%%%%%%%%%%%%%%%%%%%%%%%%%%%%%%%%%%%%%%%%%%%%%%%%%%%%
% \newcommand{\Alogon}{\mbox{\fontfamily{ptm}\selectfont {\large \selectfont A} \hspace{-1.2ex} {\large \selectfont L} \hspace{-2.3ex} \raisebox{0.45ex}{ {\footnotesize \selectfont O} } \hspace{-1.80ex} {\large \selectfont G} \hspace{-1.80ex} \raisebox{-0.33ex}{ {\large \selectfont O} } \hspace{-1.8ex} {\large \selectfont N}}}


\usepackage{graphicx}
\setlength\textwidth{6.5in}
\setlength\textheight{9.0in}
% \newcommand{\bd}{\begin{description}}
% \newcommand{\ed}{\end{description}}
\begin{document}
\title{Accelerating Decision Trees with SIMD}
\author{Tara Mirmira and Ramesh Subramonian}
\maketitle
\thispagestyle{fancy}
\lhead{}
\chead{}
\rhead{}
\lfoot{}
\cfoot{Decision Trees}
\rfoot{{\small \thepage}}

\begin{abstract}
In this paper we show the benefits and limitations of using SIMD techniques to
  accelerate decision trees, in both the building phase and the inference phase. 
\end{abstract}

\section{Introduction}

\TBC

\subsection{ISPC}

We chose to use ISPC for ``SIMD-fication''. The Intel® Implicit SPMD Program
Compiler (Intel® ISPC) is a compiler for writing SPMD (single program multiple
data) programs to run on the CPU and GPU.
Our reasons for doing so are best stated by the documentation itself
{\em 
\be
\item 
Build a variant of the C programming language that delivers good performance to performance-oriented programmers who want to run SPMD programs on CPUs and GPUs.
\item 
Provide a thin abstraction layer between the programmer and the hardware--in particular, to follow the lesson from C for serial programs of having an execution and data model where the programmer can cleanly reason about the mapping of their source program to compiled assembly language and the underlying hardware.
\item 
Harness the computational power of the Single Program, Multiple Data (SIMD) vector units without the extremely low-programmer-productivity activity of directly writing intrinsics.
\item 
Explore opportunities from close-coupling between C/C++ application code and SPMD ispc code running on the same processor--lightweight function calls between the two languages, sharing data directly via pointers without copying or reformatting, etc.
\ee
}

\subsection{Inferencing}

Given a decision tree, \(T\), and a set of data points \(X = \{x_1, x_2,
\ldots\}\), the goal is to create \(Y = \{y_1, y_2, \ldots\}\) where \(y_i\) is
the leaf to which \(x_i\) is assigned.



\subsubsection{Random Forests}

\bibliographystyle{alpha}
\bibliography{../../DOC/Q_PAPER/ref}
\end{document}
